%\documentclass{abnt}

%\usepackage[latin1]{inputenc}
%\usepackage[brazil]{babel}
%\usepackage{amsthm,amsfonts}

%\usepackage{graphicx}
%\usepackage{color}
%\usepackage{multirow}
%\usepackage{empheq,amsmath}

\documentclass[12pt,a4paper,ruledheader]{abnt}
\usepackage[center]{caption}
\usepackage[brazil]{babel}
\usepackage[latin1]{inputenc}
\usepackage{indentfirst}

\newcommand{\nome}{LUIZ GUILHERME CASTILHO MARTINS}

\newcommand{\titulotcc}{DESENVOLVIMENTO DE T�CNICAS DE PARALELIZA��O DE C�DIGO}
\newcommand{\titulotccen}{DESIGNING TECHNIQUES FOR CODE PARALELLIZATION}

\usepackage[pdftex]{graphicx}
\usepackage{latexsym}
\usepackage{ae}
\usepackage{hyperref}
%\usepackage[alf]{abntcite}
\usepackage{lscape}
\renewcommand{\rmdefault}{aess}
\renewcommand{\sfdefault}{aess}

\usepackage{amsmath}
\usepackage{xtab}
\usepackage{multirow}
\numberwithin{figure}{chapter}
\numberwithin{table}{chapter}

 %comando para que a lista de figura saia com numero dos capitulos
\numberwithin{figure}{chapter}
%  \numberwithin{algorithm}{chapter}

% %altera a fonte do documento
%  \renewcommand{\rmdefault}{aess}
%  \renewcommand{\sfdefault}{aess}
%  \renewcommand{\familydefault}{\sfdefault}


% altera o tamanho da fonte dos capitulos
%\renewcommand{\ABNTchaptersize}{\huge}

% altera o tamanho da fonte dos capitulos
\renewcommand{\ABNTchaptersize}{\LARGE}

% altera  o tipo da fonte dos capitulos
\renewcommand{\ABNTchapterfont}{\bfseries}

% altera o tamanho da fonte da secao
\renewcommand{\ABNTsectionfont}{\bf \large}

 
%Comandos definidos pelo usuario
\newcommand{\tamanhogrande}{\fontsize{16pt}{20pt}\selectfont}
\newcommand{\tamanhomedio}{\fontsize{14pt}{18pt}\selectfont}
\newcommand{\tamanhonormal}{\fontsize{12pt}{16pt}\selectfont}

\setlength{\parindent}{3cm}
\setlength{\listparindent}{3cm}

\begin{document}

	 %%%%%%%%%%%%%%%%%%%%%%%%%%%%%%%%%%%%%
%%   -capa
%%%%%%%%%%%%%%%%%%%%%%%%%%%%%%%%%%%%%
\thispagestyle{empty}
		\begin{figure}[htb]
					\centering \includegraphics[width=14cm]{1-capa/imagens/logo-uel.png}
		\end{figure}

		\begin{figure}[htb]
					\centering \includegraphics[scale=0.7]{1-capa/imagens/barra-capa.png}
		\end{figure}

		\begin{center}
            {\bf \large \nome }
    \end{center} 
    \vspace*{\stretch{4}}
				\begin{center}
                {\large \textbf {\titulotcc}} \\ \vspace{2ex}
				\end{center}

		\vspace*{\stretch{6}}

		\begin{figure}[htb]
				\centering \includegraphics[scale=0.7]{1-capa/imagens/barra-capa.png}
		\end{figure}

		\begin{center}
				{\tamanhonormal Londrina - PR \\ 2013}
		\end{center}



%%%%%%%%%%%%%%%%%%%%%%%%%%%%%%%%%%%%%
%%   FICHA CATALOGR�FICA
%%%%%%%%%%%%%%%%%%%%%%%%%%%%%%%%%%%%%
\newpage
		\noindent $ $

		\vspace{5.0in}

		\begin{center}
					\begin{tabular}{|p{14cm}|}
								\hline
								A FICHA CATALOGR�FICA DEVER�/PODER� SER SOLICITADA NA BIBLIOTECA CENTRAL
								\\ \hline
					\end{tabular}
		\end{center}



%%%%%%%%%%%%%%%%%%%%%%%%%%%%%%%%%%%%%
%%   Contracapa
%%%%%%%%%%%%%%%%%%%%%%%%%%%%%%%%%%%%%
\newpage
		\thispagestyle{empty}

		\begin{center}
				{\large \nome\\}
				\vspace{3.0in}
        {\large \textbf {\titulotcc}} \\ 
				 \vspace{2ex}

				\vspace{0.5in}
				\begin{flushright}
							\parbox{3.4in}{{\singlespacing Trabalho de Conclus�o de Curso  apresentado � Universidade Estadual de Londrina como parte dos
                      requisitos para obten��o do t�tulo de bacharel em Ci�ncia da Computa��o.}
							{\singlespacing Orientador: Prof. Dr. Wesley Attrot.}}
				\end{flushright}
				\vfill
				\vfill
				{\tamanhonormal Londrina - PR \\ 2013}
		\end{center}



%%%%%%%%%%%%%%%%%%%%%%%%%%%%%%%%%%%%%
%%   Folha de rosto
%%%%%%%%%%%%%%%%%%%%%%%%%%%%%%%%%%%%%
\newpage
		\thispagestyle{empty}

		\begin{center}
				{\large \nome}

				\vspace{0.8in}
        {\large \textbf {\titulotcc}} \\ 
				 \vspace{2ex}

				\vspace{0.5in}
				\begin{flushright}
							\parbox{3.4in}{{\singlespacing Trabalho de Conclus�o de Curso  apresentado �  Universidade Estadual de Londrina como parte dos
requisitos para obten��o do t��tulo bacharel em Ci�ncia da Computa��o.}
							{\singlespacing Orientador: Prof. Dr. Wesley Attrot.}
							}
				\end{flushright}

				\vspace{0.15in}

				\begin{flushright}
						\parbox{3.50in}{\bf{COMISS�O EXAMINADORA}}
				\end{flushright}

				\vspace{0.3in}

				\begin{flushright}
						\begin{tabular}{c}
									\hline
									Prof. Dr. Wesley Attrot\\
									Universidade Estadual de Londrina - UEL \\ \\ \\

									\hline
									Prof. Dr. -------------------------- \\
									Universidade Estadual de Londrina - UEL \\ \\ \\

									\hline
									Prof. Ms. -------------------------  \\
									Universidade Estadual de Londrina - UEL \\ \\ \\

									Londrina, -- de ---------- 2013.
						\end{tabular}
			\end{flushright}
		\end{center}



%%%%%%%%%%%%%%%%%%%%%%%%%%%%%%%%%%%%%
%%   Dedicat�ria
%%%%%%%%%%%%%%%%%%%%%%%%%%%%%%%%%%%%%
\newpage
\
\vfill
\begin{flushright}
		\hfill \textit{A todos aqueles que \\ me apoiaram e contribu�ram \\ para conclus�o deste trabalho.}
\end{flushright}
\vspace*{1cm}
\clearpage



%%%%%%%%%%%%%%%%%%%%%%%%%%%%%%%%%%%%%
%%   Agradecimentos
%%%%%%%%%%%%%%%%%%%%%%%%%%%%%%%%%%%%%
\newpage
\chapter*{AGRADECIMENTOS}
		\begin{trivlist}  \itemsep 2ex  \normalsize
					\item Agrade�o primeiramente aos meus pais por todo apoio que eles me deram durante a gradua��o.

		\end{trivlist}


%%%%%%%%%%%%%%%%%%%%%%%%%%%%%%%%%%%%%
%%   Epigrafe
%%%%%%%%%%%%%%%%%%%%%%%%%%%%%%%%%%%%%
\newpage
\
\vfill
\begin{flushright}
		%\hfill \textit{"texto da ep�grafe"} \\
		%{\bf autor da ep�grafe}
		\hfill \textit{"O importante � ganhar. Tudo e sempre. \\Essa hist�ria que o importante � competir n�o passa de demagogia."} \\
        {\bf Ayrton Senna}
\end{flushright}
\vspace*{1cm}


%%%%%%%%%%%%%%%%%%%%%%%%%%%%%%%%%%%%%
%%   Resumo
%%%%%%%%%%%%%%%%%%%%%%%%%%%%%%%%%%%%%
\newpage
\singlespacing
		\noindent MARTINS, Luiz Guilherme Castilho {\bf \titulotcc}. 2013. 173 f. TCC - Universidade Estadual de Londrina, Londrina. 2013. \\
		\begin{center}
					{\bf {\Large RESUMO}}
		\end{center}

		\noindent Resumo.
    \newline

		\noindent{\bf Palavras-Chave:} palavra1, palavra2.


%%%%%%%%%%%%%%%%%%%%%%%%%%%%%%%%%%%%%
%%   Abstract
%%%%%%%%%%%%%%%%%%%%%%%%%%%%%%%%%%%%%
\newpage
\singlespacing
    \noindent MARTINS, Luiz Guilherme Castilho {\bf \titulotccen}. 2013. 173 f. TCC - Universidade Estadual de Londrina, Londrina. 2013. \\
		\begin{center}
					{\bf {\Large ABSTRACT}}
		\end{center}

		\noindent Abstract.
    \newline

		\noindent{\bf Palavras-Chave:}  word1, word2.


	\sumario
	\onehalfspacing
  \listoffigures
	\listoftables	
	\chapter*{Lista de Siglas e Abreviaturas}
\noindent 
\begin{tabular}{l c p{.85\linewidth}}
        % I
        {\bf IW} & - & Instruction Word \\
        
        % M
        {\bf MISD} & - & Multiple Instruction Single Data \\
        {\bf MIMD} & - & Multiple Instruction Multiple Data \\

        % S
        {\bf SISD} & - & Single Instruction Single Data \\
        {\bf SIMD} & - & Single Instruction Multiple Data \\

        %V
        {\bf VLIW} & - & Very Long Instruction Word \\
\end{tabular}

%			\noindent \begin{tabular}{l c p{.85\linewidth}}
%						
%						%Exemplo
%						% A
%						{\bf AES}	&-	& Advanced Encryption Standard \\
%						
%						% B
%						{\bf BE} &- & Broadband Engine \\
%						
%						% C
%						{\bf CBEA} &- & Cell Broadband Engine Architecture \\
%						
%			\end{tabular}
%			
%			
%			
%			\noindent \begin{tabular}{l c p{.85\linewidth}}
%						
%						% P
%						{\bf PADIS} &- & Programa de Apoio ao Desenvolvimento Tecnol�gico da Ind�stria de Semicondutores \\
%						
%						% Q
%						{\bf QPI}	&-	& Quick Path Interconnect \\
%						
%						% R
%						{\bf RAM} &- & Random Access Memory \\
%						{\bf RISC}	&-	& Reduced Instruction Set Computer \\
%						
%						% {\bf SIGLA1}	&-	& Significado \\
%						% {\bf SIGLA1}	&-	& Significado \\
%						
%			\end{tabular}

	\tableofcontents
	
	% vim: set ai tw=80 fileencoding=utf8: 
%-------------------------------------------------------------------------------
\chapter{Introdução}

Diante da  dificuldade de se melhorar processadores \textit{single-core} devido ao alto consumo de energia e as altas
temperaturas, a indústria de processadores adotou como solução à esses
problemas, o desenvolvimento de processadores \textit{multi-core},
além disso também considerou-se o poder de processamento quando se utilizam vários processadores \textit{single-core} 
trabalhando simultaneamente, os quais oferecem grande desempenho, maior eficiência energética e menor custo.
Porém as aplicações desenvolvidas para a arquitetura \textit{single-core} não utilizam-se do poder computacional dos processadores \textit{multi-core}.
Para que as aplicações possam usufruir do alto desempenho que estes processadores oferecem, as aplicações devem ser 
divididas em múltiplas partes para que possam ser executadas em paralelo.

A paralelização de código permite ao programador resolver problemas com maior eficiência, porém projetar e codificar um 
programa paralelo continua sendo uma tarefa difícil, uma vez que além de dividi-lo em pequenas tarefas, deve-se considerar a 
concorrência entre elas, as dependências de dados, entre outros fatores que dificultam o trabalho de paralelização 



  % vim: set ai tw=80: 
%-------------------------------------------------------------------------------
\chapter{Taxonomia de Flynn}

\noindent Taxonomia de Flynn, proposta em 1966 \cite{Flynn:1966} por Michael 
Flynn e expandida em 1972 \cite{Flynn:1972}, � uma das formas de classificar o 
paralelismo dispon�vel no processador.  

Taxonomia de Flynn utiliza o conceito de sequ�ncia de objetos ou a��es, que s�o
chamados de \textit{Stream}. Flynn introduziu dois tipos de \textit{Stream}, o 
\textit{Stream} de instru��o e tamb�m o \textit{Stream} de dados. 

O \textit{Stream} de instru��o consiste em uma sequ�ncia de instru��es. 
Uma instru��o ou \textit{instruction word (IW)} � uma cadeia de 0's e 1's que 
representa a menor opera��o vis�vel ao programador e que ser� executada pelo 
processador. Uma instru��o pode conter uma ou mais opera��es, devido a isso
alguns autores utilizam \textit{instruction} para instru��es que contenham 
apenas uma opera��o e \textit{instruction word} para instru��es que contenham 
mais de uma opera��o.

\begin{comment}
Processadores escalares (\textit{scalar processors}) e processadores
superescalares (\textit{superscalar processors}) executam uma ou mais
\textit{instructions} por ciclo de \textit{clock} da m�quina. 
\end{comment}

Existem no entanto quatro combina��es de \textit{Streams} que descrevem as 
arquiteturas de computadores mais comuns \cite{Flynn:1996}:

\begin{enumerate}
        \item \textbf{SISD:} \textit{Single Instruction, Single Data}
        \item \textbf{SIMD:} \textit{Single Instruction, Multiple Data}
        \item \textbf{MISD:} \textit{Multiple Instruction, Single Data}
        \item \textbf{MIMD:} \textit{Multiple Instruction, Multiple Data}
\end{enumerate}

Cada combina��o de \textit{Stream} caracteriza uma classe de arquitetura 
e cada classe possui seus tipos de paralelismo.

\subsection{SISD: Single Instruction, Single Data}

\noindent A classe de arquiteturas de processadores \textit{SISD}, inclui a 
maior parte dos processadores utilizados nos dias de hoje, os processadores 
\textit{single-core}, embora os programadores n�o percebam o paralelismo 
inerente destes processadores, muita concorr�ncia pode estar presente.  

Em 1966 Flynn cita o \textit{Pipeline} como uma forma de se obter concorr�ncia 
nos processadores \textit{SISD}, embora ele considere a decodifica��o das 
in�meras \textit{instructions} como sendo um \textit{bottleneck}, devido a 
tecnologia da �poca. Nos dias de hoje, grande parte dos dos processadores 
utilizam-se de \textit{Pipeline} assim como tamb�m se aproveitam de alguma forma 
de m�ltiplas \textit{instructions}.

A concorr�ncia em processadores \textit{SISD} s�o explorados durante a execu��o,
assim realizando mais de uma opera��o por ciclo de \textit{clock} da m�quina 
do \textit{Stream} de instru��es.

A quantidade e o tipo de paralelismo poss�vel em processadores \textit{SISD}
� determinada por quatro fatores principais:

\begin{enumerate}
        \item O n�mero de opera��es que podem ser executadas concorrentemente.
        \item A forma como as opera��es ser�o arranjadas para execu��o,
                podendo ser estaticamente, dinamicamente ou at� mesmo das duas
                formas.
        \item A ordem em que as opera��es s�o colocadas e retiradas em rela��o
                a ordem original do programa.
        \item A maneira em o processador ir� tratar cada exce��o, podendo ser 
                preciso, impreciso ou das duas maneiras.
\end{enumerate}


\subsubsection{Processadores Escalares}


\subsubsection{Processadores Superescalares}

\subsubsection{Processadores VLIW}

\subsection{SIMD: Single Instruction, Multiple Data}

\subsection{MISD: Multiple Instruction, Single Data}

\subsection{MIMD: Multiple Instruction, Multiple Data}

	
	%\input{[4]_Conclusao/Conclusao}
	
  %\bibliographystyle{abnt-alf}
	\bibliographystyle{plain}
	% refbi � o nome do arquivo .bib
	\bibliography{3-taxonomia-flynn/taxonomia-flynn} 

	%Gloss�rio (se necess�rio)
	%\include{./glossario}
	%\input{[5]_Referencia/Bibliografia}
	%\bibliographystyle{abnt-alf}
	%\bibliography{[5]_Referencia/Bibli} 
	
	%Ap�ndices (se necess�rio)
	%\include{[6]_Apendice/Apendice_A}
	%\include{[7]_Anexos/Anexo_A}
	
\end{document}
