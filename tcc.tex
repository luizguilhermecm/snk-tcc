%\documentclass{abnt}

%\usepackage[latin1]{inputenc}
%\usepackage[brazil]{babel}
%\usepackage{amsthm,amsfonts}

%\usepackage{graphicx}
%\usepackage{color}
%\usepackage{multirow}
%\usepackage{empheq,amsmath}

\documentclass[12pt,a4paper,ruledheader]{abnt}
\usepackage[center]{caption}
\usepackage[brazil]{babel}

%Pacote para acentua��o 
%\usepackage[latin1]{inputenc}
\usepackage[utf8x]{inputenc}
\usepackage{indentfirst}

\usepackage[pdftex]{graphicx}
\usepackage{latexsym}
\usepackage{ae}
\usepackage{hyperref}
\usepackage[alf]{abntcite}
\usepackage{lscape}
\renewcommand{\rmdefault}{aess}
\renewcommand{\sfdefault}{aess}

\usepackage{amsmath}
\usepackage{xtab}
\usepackage{multirow}
\numberwithin{figure}{chapter}
\numberwithin{table}{chapter}

 %comando para que a lista de figura saia com numero dos capitulos
\numberwithin{figure}{chapter}
%  \numberwithin{algorithm}{chapter}

% %altera a fonte do documento
%  \renewcommand{\rmdefault}{aess}
%  \renewcommand{\sfdefault}{aess}
%  \renewcommand{\familydefault}{\sfdefault}

%\renewcommand{\ABNTchaptersize}{\huge}       % altera o tamanho da fonte dos capitulos
 \renewcommand{\ABNTchaptersize}{\LARGE}       % altera o tamanho da fonte dos capitulos
 \renewcommand{\ABNTchapterfont}{\bfseries}    % altera  o tipo da fonte dos capitulos
 \renewcommand{\ABNTsectionfont}{\bf \large}   % altera o tamanho da fonte da secao
 
%Comandos definidos pelo usuario
\newcommand{\tamanhogrande}{\fontsize{16pt}{20pt}\selectfont}
\newcommand{\tamanhomedio}{\fontsize{14pt}{18pt}\selectfont}
\newcommand{\tamanhonormal}{\fontsize{12pt}{16pt}\selectfont}

\setlength{\parindent}{3cm}
\setlength{\listparindent}{3cm}

\begin{document}

	% \input{1Capa/Capa}

	%\sumario
	\onehalfspacing
  %	\listoffigures
	% \listoftables	
	% \chapter*{Lista de Siglas e Abreviaturas}
	\noindent 
	\begin{tabular}{l c p{.85\linewidth}}
		{\bf GPS} &- Global Positioning System
	\end{tabular}

%			\noindent \begin{tabular}{l c p{.85\linewidth}}
%						
%						%Exemplo
%						% A
%						{\bf AES}	&-	& Advanced Encryption Standard \\
%						
%						% B
%						{\bf BE} &- & Broadband Engine \\
%						
%						% C
%						{\bf CBEA} &- & Cell Broadband Engine Architecture \\
%						
%			\end{tabular}
%			
%			
%			
%			\noindent \begin{tabular}{l c p{.85\linewidth}}
%						
%						% P
%						{\bf PADIS} &- & Programa de Apoio ao Desenvolvimento Tecnol�gico da Ind�stria de Semicondutores \\
%						
%						% Q
%						{\bf QPI}	&-	& Quick Path Interconnect \\
%						
%						% R
%						{\bf RAM} &- & Random Access Memory \\
%						{\bf RISC}	&-	& Reduced Instruction Set Computer \\
%						
%						% {\bf SIGLA1}	&-	& Significado \\
%						% {\bf SIGLA1}	&-	& Significado \\
%						
%			\end{tabular}
	% \tableofcontents
	
	%\input{2Introducao/Introducao}

  % vim: set ai tw=80: 
%-------------------------------------------------------------------------------
\chapter{Taxonomia de Flynn}

\noindent Taxonomia de Flynn, proposta em 1966 \cite{Flynn:1966} por Michael 
Flynn e expandida em 1972 \cite{Flynn:1972}, � uma das formas de classificar o 
paralelismo dispon�vel no processador.  

Taxonomia de Flynn utiliza o conceito de sequ�ncia de objetos ou a��es, que s�o
chamados de \textit{Stream}. Flynn introduziu dois tipos de \textit{Stream}, o 
\textit{Stream} de instru��o e tamb�m o \textit{Stream} de dados. 

O \textit{Stream} de instru��o consiste em uma sequ�ncia de instru��es. 
Uma instru��o ou \textit{instruction word (IW)} � uma cadeia de 0's e 1's que 
representa a menor opera��o vis�vel ao programador e que ser� executada pelo 
processador. Uma instru��o pode conter uma ou mais opera��es, devido a isso
alguns autores utilizam \textit{instruction} para instru��es que contenham 
apenas uma opera��o e \textit{instruction word} para instru��es que contenham 
mais de uma opera��o.

\begin{comment}
Processadores escalares (\textit{scalar processors}) e processadores
superescalares (\textit{superscalar processors}) executam uma ou mais
\textit{instructions} por ciclo de \textit{clock} da m�quina. 
\end{comment}

Existem no entanto quatro combina��es de \textit{Streams} que descrevem as 
arquiteturas de computadores mais comuns \cite{Flynn:1996}:

\begin{enumerate}
        \item \textbf{SISD:} \textit{Single Instruction, Single Data}
        \item \textbf{SIMD:} \textit{Single Instruction, Multiple Data}
        \item \textbf{MISD:} \textit{Multiple Instruction, Single Data}
        \item \textbf{MIMD:} \textit{Multiple Instruction, Multiple Data}
\end{enumerate}

Cada combina��o de \textit{Stream} caracteriza uma classe de arquitetura 
e cada classe possui seus tipos de paralelismo.

\subsection{SISD: Single Instruction, Single Data}

\noindent A classe de arquiteturas de processadores \textit{SISD}, inclui a 
maior parte dos processadores utilizados nos dias de hoje, os processadores 
\textit{single-core}, embora os programadores n�o percebam o paralelismo 
inerente destes processadores, muita concorr�ncia pode estar presente.  

Em 1966 Flynn cita o \textit{Pipeline} como uma forma de se obter concorr�ncia 
nos processadores \textit{SISD}, embora ele considere a decodifica��o das 
in�meras \textit{instructions} como sendo um \textit{bottleneck}, devido a 
tecnologia da �poca. Nos dias de hoje, grande parte dos dos processadores 
utilizam-se de \textit{Pipeline} assim como tamb�m se aproveitam de alguma forma 
de m�ltiplas \textit{instructions}.

A concorr�ncia em processadores \textit{SISD} s�o explorados durante a execu��o,
assim realizando mais de uma opera��o por ciclo de \textit{clock} da m�quina 
do \textit{Stream} de instru��es.

A quantidade e o tipo de paralelismo poss�vel em processadores \textit{SISD}
� determinada por quatro fatores principais:

\begin{enumerate}
        \item O n�mero de opera��es que podem ser executadas concorrentemente.
        \item A forma como as opera��es ser�o arranjadas para execu��o,
                podendo ser estaticamente, dinamicamente ou at� mesmo das duas
                formas.
        \item A ordem em que as opera��es s�o colocadas e retiradas em rela��o
                a ordem original do programa.
        \item A maneira em o processador ir� tratar cada exce��o, podendo ser 
                preciso, impreciso ou das duas maneiras.
\end{enumerate}


\subsubsection{Processadores Escalares}


\subsubsection{Processadores Superescalares}

\subsubsection{Processadores VLIW}

\subsection{SIMD: Single Instruction, Multiple Data}

\subsection{MISD: Multiple Instruction, Single Data}

\subsection{MIMD: Multiple Instruction, Multiple Data}

	
	%\input{[3]_Desenvolvimento/[Cap_1]_A_HISTORIA_DA_CRIACAO_E_A_EVOLUCAO_DOS_PROCESSADORES/Desenvolvimento}
	
	%\input{[4]_Conclusao/Conclusao}
	
	%\bibliographystyle{abnt-alf}
	% refbi � o nome do arquivo .bib
	%\bibliography{[5]_Referencia/Bibli} 

	%Gloss�rio (se necess�rio)
	%\include{./glossario}
	%\input{[5]_Referencia/Bibliografia}
	%\bibliographystyle{abnt-alf}
	%\bibliography{[5]_Referencia/Bibli} 
	
	%Ap�ndices (se necess�rio)
	%\include{[6]_Apendice/Apendice_A}
	%\include{[7]_Anexos/Anexo_A}
	
\end{document}
