% vim: set ai tw=80 fileencoding=utf8: 
%-------------------------------------------------------------------------------
\documentclass[
  % -- opções da classe memoir --
	12pt,				% tamanho da fonte
	openright,			% capítulos começam em pág ímpar (página vazia se preciso)
  %twoside,			% para impressão em verso e anverso. Oposto a oneside
  oneside,			% oposto de twoside, para nao gerar paginas 
	a4paper,			% tamanho do papel. 
	% -- opções da classe dc-uel --
	tcc,			% tipo do trabalho (opções: tcc, dissertacao, qualificacaoms)
	]{dc-uel}


% ---
% PACOTES
% ---

% ---
% Pacotes fundamentais 
% ---
\usepackage[T1]{fontenc}		% Selecao de codigos de fonte.
\usepackage[utf8]{inputenc}		% Codificacao do documento
\usepackage{graphicx}			% Inclusão de gráficos
% ---
		
% ---
% Pacotes adicionais, usados apenas no âmbito do Modelo Canônico do abnteX2
% ---
\usepackage{lipsum}				% para geração de dummy text
% ---

% ---
% Informações de dados para CAPA, FOLHA DE ROSTO e outros elementos
% ---
\titulo{Desenvolvimento de Técnicas de Paralelização de Código}
\tituloingles{Designing Techiniques for Code Paralellization}
\palavraschave{palavra chave1. palavra chave2.}
\palavraschaveingles{word 1. key word2.}
\autor{Luiz Guilherme Castilho Martins}
\citacaoautor{MARTINS, L. G. C.}
\data{2013}

\diadefesa{24 de novembro}
\orientador{Prof. Dr. Wesley Attrot} % É membro nato e presidente da Banca Examinadora
\membrobancadois{Prof. Dr. Segundo Membro da Banca}
\instmembrobancadois{Universidade Estadual de Londrina}
\membrobancatres{Prof. Msc. Terceiro Membro da Banca}
\instmembrobancatres{Universidade Estadual de Londrina}
% \membrobancaquatro{Prof. Esp. Quarto Membro da Banca}
% \instmembrobancaquatro{Universidade/Instituição do Quarto Membro da Banca}

% ---
% compila o indice
% ---
\makeindex
% ---

% ----
% Início do documento
% ----
\begin{document}

% Retira espaço extra obsoleto entre as frases.
\frenchspacing 

% ----------------------------------------------------------
% ELEMENTOS PRÉ-TEXTUAIS
% ----------------------------------------------------------
% \pretextual


%  % vim: set ai tw=80 fileencoding=latin1: 
%-------------------------------------------------------------------------------

\thispagestyle{empty}
		\begin{figure}[htb]
					\centering \includegraphics[width=14cm]{1-capa/imagens/logo-uel.png}
		\end{figure}

		\begin{figure}[htb]
					\centering \includegraphics[scale=0.7]{1-capa/imagens/barra-capa.png}
		\end{figure}

		\begin{center}
            {\bf \large \nome }
    \end{center} 
    \vspace*{\stretch{4}}
				\begin{center}
                {\large \textbf {\titulotcc}} \\ \vspace{2ex}
				\end{center}

		\vspace*{\stretch{6}}

		\begin{figure}[htb]
				\centering \includegraphics[scale=0.7]{1-capa/imagens/barra-capa.png}
		\end{figure}

		\begin{center}
				{\tamanhonormal Londrina - PR \\ 2013}
		\end{center}

%  \input{1-capa/1-folha-de-rosto}
%  \input{1-capa/2-errata}
%  % vim: set ai tw=80 fileencoding=utf8: 
%-------------------------------------------------------------------------------

\includepdf{folha-aprovacao}

%  % vim: set ai tw=80 fileencoding=latin1: 
%-------------------------------------------------------------------------------


\newpage
\
\vfill
\begin{flushright}
		\hfill \textit{A todos aqueles que \\ 
                        me apoiaram e contribu�ram \\ 
                        para conclus�o deste trabalho.}
\end{flushright}
\vspace*{1cm}
\clearpage


%  % vim: set ai tw=80 fileencoding=utf8: 
%-------------------------------------------------------------------------------
\begin{agradecimentos}
Agradeço primeiramente aos meus irmãos que estiveram por perto durante toda
a graduação e meus pais que me apoiaram e incentivaram 
em tudo e sempre.
        
Professor Dr. Wesley Attrot, que propôs este tema desafiador e me motivou e 
incentivou durante a realização deste trabalho.

Aos amigos: 
Arthur Coutinho que me ajudou muito neste e em outros desafios. 
Beatriz que todos temem. 
Breno Kusunoki que sempre diz que horas são.
Mateus Piveta. 
Clayton Kuwabara por ser mac fag. 
Hélio que me ensiou que não está fácil nem para ela. 
Randal que me ajudou a corrigir este trabalho e é bobo. 
Ernesto que adora harumaki.
E todos os amigos que fiz durante a graduação, todas as horas de estudo
e brincadeiras que tivemos juntos.

Agradeço a Bijuzinho, Gab, Morgana, Janja, Princesa Léia, e Maria, responsáveis
por momentos de descontração durante a graduação.

Todos os professores que de uma forma ou outra, muito me ensinaram.

Agredeço também todos os funcionários da UEL, que no exercício de suas funções,
ajudaram a tornar a graduação mais agradável.
\end{agradecimentos}

%  % vim: set ai tw=80 fileencoding=latin1: 
%-------------------------------------------------------------------------------

\newpage
\
\vfill
\begin{flushright}
		%\hfill \textit{"texto da ep�grafe"} \\
		%{\bf autor da ep�grafe}
		\hfill \textit{"O importante � ganhar. Tudo e sempre. \\
                        Essa hist�ria que o importante � competir n�o passa de 
                        demagogia."} \\
        {\bf Ayrton Senna}
\end{flushright}
\vspace*{1cm}


%  % vim: set ai tw=80 fileencoding=utf8: 
%-------------------------------------------------------------------------------
\begin{resumo}
        Paralelização de código permite ao programador a oportunidade de criar algoritmos para resolver problemas com maior eficiência.
        Programas são ditos paralelos quando existem duas ou mais ações executando simultaneamente em diferentes unidades de processamento.
        O objetivo deste trabalho é estudar técnicas de paralelização de código e tentar desenvolver uma nova otimização ou técnica de paralelização.
\end{resumo}

%  % vim: set ai tw=80 fileencoding=utf8: 
%-------------------------------------------------------------------------------
\begin{Abstract}
 This is the english abstract. The Abstract in English should be faithful to the
 Resumo in Portuguese, but not a literal translation.
\end{Abstract}


%
%% ---
%% inserir lista de ilustrações
%% ---
%\pdfbookmark[0]{\listfigurename}{lof}
%\listoffigures*
%\cleardoublepage
%% ---
%
%% ---
%% inserir lista de tabelas
%% ---
%\pdfbookmark[0]{\listtablename}{lot}
%\listoftables*
%\cleardoublepage
%% ---
%
 % vim: set ai tw=80 fileencoding=utf8: 
%-------------------------------------------------------------------------------
\begin{siglas}
        % C
        \item[CPU] Central Processing Unit

        % I
        \item[IW] Instruction Word 

        % M
        \item[MISD] Multiple Instruction Single Data
        \item[MIMD] Multiple Instruction Multiple Data

        % S
        \item[SISD] Single Instruction Single Data
        \item[SIMD] Single Instruction Multiple Data

        % V
        \item[VLIW] Very Long Instruction Word
\end{siglas}

%
%% vim: set ai tw=80 fileencoding=utf8: 
%-------------------------------------------------------------------------------
\begin{simbolos}
  \item[$ \delta $] Letra grega delta 
\end{simbolos}

%
%
% ---
% inserir o sumario
% ---
 \pdfbookmark[0]{\contentsname}{toc}
 \tableofcontents*
 \cleardoublepage
% ---
%
%\textual
%
%% vim: set ai tw=80 fileencoding=utf8: 
%-------------------------------------------------------------------------------
\chapter{Introdução}

Diante da  dificuldade de se melhorar processadores \textit{single-core} devido ao alto consumo de energia e as altas
temperaturas, a indústria de processadores adotou como solução à esses
problemas, o desenvolvimento de processadores \textit{multi-core},
além disso também considerou-se o poder de processamento quando se utilizam vários processadores \textit{single-core} 
trabalhando simultaneamente, os quais oferecem grande desempenho, maior eficiência energética e menor custo.
Porém as aplicações desenvolvidas para a arquitetura \textit{single-core} não utilizam-se do poder computacional dos processadores \textit{multi-core}.
Para que as aplicações possam usufruir do alto desempenho que estes processadores oferecem, as aplicações devem ser 
divididas em múltiplas partes para que possam ser executadas em paralelo.

A paralelização de código permite ao programador resolver problemas com maior eficiência, porém projetar e codificar um 
programa paralelo continua sendo uma tarefa difícil, uma vez que além de dividi-lo em pequenas tarefas, deve-se considerar a 
concorrência entre elas, as dependências de dados, entre outros fatores que dificultam o trabalho de paralelização 


%
%% vim: set ai tw=80: 
%-------------------------------------------------------------------------------
\chapter{Taxonomia de Flynn}

\noindent Taxonomia de Flynn, proposta em 1966 \cite{Flynn:1966} por Michael 
Flynn e expandida em 1972 \cite{Flynn:1972}, � uma das formas de classificar o 
paralelismo dispon�vel no processador.  

Taxonomia de Flynn utiliza o conceito de sequ�ncia de objetos ou a��es, que s�o
chamados de \textit{Stream}. Flynn introduziu dois tipos de \textit{Stream}, o 
\textit{Stream} de instru��o e tamb�m o \textit{Stream} de dados. 

O \textit{Stream} de instru��o consiste em uma sequ�ncia de instru��es. 
Uma instru��o ou \textit{instruction word (IW)} � uma cadeia de 0's e 1's que 
representa a menor opera��o vis�vel ao programador e que ser� executada pelo 
processador. Uma instru��o pode conter uma ou mais opera��es, devido a isso
alguns autores utilizam \textit{instruction} para instru��es que contenham 
apenas uma opera��o e \textit{instruction word} para instru��es que contenham 
mais de uma opera��o.

\begin{comment}
Processadores escalares (\textit{scalar processors}) e processadores
superescalares (\textit{superscalar processors}) executam uma ou mais
\textit{instructions} por ciclo de \textit{clock} da m�quina. 
\end{comment}

Existem no entanto quatro combina��es de \textit{Streams} que descrevem as 
arquiteturas de computadores mais comuns \cite{Flynn:1996}:

\begin{enumerate}
        \item \textbf{SISD:} \textit{Single Instruction, Single Data}
        \item \textbf{SIMD:} \textit{Single Instruction, Multiple Data}
        \item \textbf{MISD:} \textit{Multiple Instruction, Single Data}
        \item \textbf{MIMD:} \textit{Multiple Instruction, Multiple Data}
\end{enumerate}

Cada combina��o de \textit{Stream} caracteriza uma classe de arquitetura 
e cada classe possui seus tipos de paralelismo.

\subsection{SISD: Single Instruction, Single Data}

\noindent A classe de arquiteturas de processadores \textit{SISD}, inclui a 
maior parte dos processadores utilizados nos dias de hoje, os processadores 
\textit{single-core}, embora os programadores n�o percebam o paralelismo 
inerente destes processadores, muita concorr�ncia pode estar presente.  

Em 1966 Flynn cita o \textit{Pipeline} como uma forma de se obter concorr�ncia 
nos processadores \textit{SISD}, embora ele considere a decodifica��o das 
in�meras \textit{instructions} como sendo um \textit{bottleneck}, devido a 
tecnologia da �poca. Nos dias de hoje, grande parte dos dos processadores 
utilizam-se de \textit{Pipeline} assim como tamb�m se aproveitam de alguma forma 
de m�ltiplas \textit{instructions}.

A concorr�ncia em processadores \textit{SISD} s�o explorados durante a execu��o,
assim realizando mais de uma opera��o por ciclo de \textit{clock} da m�quina 
do \textit{Stream} de instru��es.

A quantidade e o tipo de paralelismo poss�vel em processadores \textit{SISD}
� determinada por quatro fatores principais:

\begin{enumerate}
        \item O n�mero de opera��es que podem ser executadas concorrentemente.
        \item A forma como as opera��es ser�o arranjadas para execu��o,
                podendo ser estaticamente, dinamicamente ou at� mesmo das duas
                formas.
        \item A ordem em que as opera��es s�o colocadas e retiradas em rela��o
                a ordem original do programa.
        \item A maneira em o processador ir� tratar cada exce��o, podendo ser 
                preciso, impreciso ou das duas maneiras.
\end{enumerate}


\subsubsection{Processadores Escalares}


\subsubsection{Processadores Superescalares}

\subsubsection{Processadores VLIW}

\subsection{SIMD: Single Instruction, Multiple Data}

\subsection{MISD: Multiple Instruction, Single Data}

\subsection{MIMD: Multiple Instruction, Multiple Data}

%	
%% vim: set ai tw=80 fileencoding=latin1: 
%-------------------------------------------------------------------------------
\chapter{Mem�ria}

\noindent A mais simples maneira de se melhorar o desempenho de um sistema �
replicar os computadores e criar uma forma destes trocarem dados.
Desta forma consegue-se aumentar o desempenho sem que seja necess�rio alterar a
\textit{CPU}.

Com o aumento cont�nuo da necessidade de desempenho em aplica��es cada vez mais
custosas a maioria dos sistemas paralelos utilizam-se de uma entre duas
t�cnologias, mem�ria distribu�da ou mem�ria compartilhada.


%-------------------------------------------------------------------------------
\subsection{Mem�ria Distribu�da}
\noindent
Mem�ria distribu�da ou \textit{distributed memory} ou \textit{shared-nothing} �
a mais simples abordagem do ponto de vista do \textit{hardware}. A premissa
desta abordagem � utilizar v�rios computadores interligados atrav�z de uma rede.

O modelo padr�o de programa��o consiste de processos separados para cada
computador que se comunicam atrav�z da troca de mensagem ou 
\textit{message passing}, o que normalmente � feito atrav�z de bibliotecas
desenvolvidas com esse prop�sito. Sendo este � o modelo mais cl�ssico de 
computa��o paralela. A forma moderna de sistemas com mem�ria distribu�da iniciou
apartir do trabalho de Seitz em 1985 \cite{Seitz:1985}.

Devido ao baixo custo de processadores voltados ao mercado consumidor e da f�cil
montagem, alguns grupos exploraram tais fatores come�aram a construir
\textit{cluster} de computadores pessoais. Tais \textit{clusters} j� chamados de
\textit{NOWs}, \textit{Network of Workstations}.
Combinando todos estes fatores com o r�pido avan�o de desempenho de computadores
pessoais e o avan�o do \textit{open-source} junto com  vers�es de sistemas 
operacionais UNIX ajudaram a difundir sistemas com tais caracteristicas. Hoje
estes sistemas s�o comumente conhecidos como \textit{Beowulfs} ou
\textit{Beowulf Cluster} devido ao projeto de Thomas Sterling e Donald Becker
realizado na NASA.

%-------------------------------------------------------------------------------
\subsection{Mem�ria Compartilhada}

%
 % vim: set ai tw=80 fileencoding=utf8: 
%-------------------------------------------------------------------------------

\chapter{Data Flow Graph}

\textit{Data Flow Graph} (DFG) ou grafo de fluxo de dados, é um modelo para 
programas que expressa a possibilidade de execução concorrênte de partes do
programa. 
Nos DFGs os nós representam operações (funções) e predicados a serem
aplicados a objetos de dados e as arestas representam a ligação entre o nó que
produz o dado ao nó que irá consumir aquele dado.
Na literatura os nós também são chamados de atores.
Assim, aspectos de controle e de dados de um programa podem ser representados 
em um único modelo integrado.

Embora muitas versões de DFGs tem sido estudadas na literatura, elas tem algumas
características em comum:

\begin{alineas}
        \item DFG é um grafo orientado onde uma aresta é um caminho que um dado
        percorre do nó produtor para o nó consumidor.
        \item Dinamicamente, o nó de um DFG aceita um ou mais dado como entrada,
        realizando computações e devolvendo os dados do retorno para suas
        saídas.
        \item Uma ação de um nó é ativada com a presença dos dados de entrada.
\end{alineas}

Os estudos em DFGs tem sido focado principalmente em três modelos bem definidos:
DFGs estáticos, DFGs dinâmicos e DFGs síncronos.

%eopc
%-------------------------------------------------------------------------------
%\section{Subcapítulo}


Texto...


 % vim: set ai tw=80 fileencoding=utf8: 
%-------------------------------------------------------------------------------

\chapter{Depêndencia}

Quando um programador escreve um programa em uma linguagem sequêncial, o
resultado esperado será obtido pela execução da primeira linha, depois a segundo
e assim em diante, considerando exceções de controles de fluxos como
\textit{loops} e ramificações. 
Uma vez que o programador especificou a ordem que ele espera que as computações 
sejam realizadas. 
Obter parelelismo de um programa respeitando a estas especificações não é
possível, uma vez que obter paralelismo implica em alterar a ordem das
operações realizadas.

Paralelizar um programa sequêncial significa encontrar uma ordem de execução
diferente da especificada e que irá sempre computar o mesmo resultado.
A programação sequêncial introduz restrições que podem ser críticas para o
resultado esperado do programa, assim para transformar um programa em paralelo é
importante encontrar as restrições menos críticas e realizar transformações para
que o programa continue retornando o resultado correto para qualquer entrada.

Neste capítudo serão apresentadas uma série de restrições, chamadas de
dependências que serão necessárias para garantir que as transformações
realizadas nos programas não afetem o resultado e o significado das computações
realizadas pelo programa.

Uma dependência é uma relação entre duas declarações no programa. Um par de
declarações $<S_1,S_2>$ está em uma relação se $S_2$ é executada depois de $S_1$
em um programa sequêncial, e deve ser executada após $S_1$ em qualquer
reodenação válida do programa se a ordem de acesso a memória será preservada.

\begin{verbatim}
S1   PI = 3.14159
S2   R = 5 
S3   AREA = PI * R * R
\end{verbatim}

%referencia: ocfma
%-------------------------------------------------------------------------------




% ----------------------------------------------------------
% ELEMENTOS PÓS-TEXTUAIS
% ----------------------------------------------------------
\postextual


% ----------------------------------------------------------
% Referências bibliográficas
% ----------------------------------------------------------
\bibliography{3-taxonomia-flynn/taxonomia-flynn} 

\end{document}
