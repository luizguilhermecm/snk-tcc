% vim: set ai tw=80 fileencoding=utf8: 
%-------------------------------------------------------------------------------

\section{Trabalhos Relacionados}

Nesta seção serão aprensentadas abordagens realizadas por outros autores que 
desenvolveram estudo de técnicas relacionadas ao conteúdo deste trabalho.
Todas as abordagens mencionadas têm como objetivo obter maior melhoria de 
desempenho.



%-------------------------------------------------------------------------------
% Exploitation of parallelism to nested loops with dependence cycles
%-------------------------------------------------------------------------------

Em \textit{"Exploitation of parallelism to nested loops with dependence
cyles"} \cite{Chang:2004}, os autores estudam a exploração de paralelismo de
\textit{loops} contendo ciclos de dependências, com foco na quebra das relações 
de dependências para então extrair paralelismo destes \textit{loops} com menos
ou nenhuma relação de dependência.

Os autores partem realizando uma revisão teórica, apresentando os principais
conceitos de dependência, assim como as principais relações de dependências, 
\textit{true-dependence}, \textit{output-dependence} e \textit{anti-dependence}.  
São também apresentados alguns conceitos como \textit{loop independent dependence}, 
\textit{loop-carried dependence}, \textit{dependence distance vector}, 
\textit{dependence distance matrix}, \textit{inter-statement} ou
\textit{intra-statement} dependences e \textit{direction vector matrix} de
\textit{loops} aninhados.

É colocado pelos autores que os ciclos de relações \textit{anti-dependence} e
\textit{true-dependence} podem ser quebradas embora \textit{true-dependence} ser
inquebrável.

A estratégia para quebrar a relação de dependência \textit{anti-dependence}
apresentada se baseia no uso de uma técnica de renomeação de variáveis
chamada \textit{sink}. 
A partir da qual é criada um variável nova que será utilizada para manter os
valores daquela envolvida na relação \textit{anti-dependence}.  
Para \textit{output-dependence} o mesmo algoritmo é apresentado.

Uma vez que os ciclos de relações de \textit{true-dependence} não pode ser
quebrados, os autores apresentam várias técnicas que premitem amenizar as
relações de dependências, o resultado deste ultimo algoritmo é especialmente
voltado para a aplicação de vetorização.

Os autores mostram que a aplicabilidade de seus algoritmos satisfizeram e
mantiveram todas as relações de dependências iniciais, assim não alterando o 
resultado final.

Os resultados obtidos utilizando \textit{vector loops} mostraram que as técnicas
propóstas é muito significativa em termos de \textit{speed-up}, a execução dos 
programas tranformados foram de 41 a 69 vezes rápidos em comparação a execução
dos programas originais.





%Em \cite{Behera:2006} os autores apresentam uma melhoria no algoritmo 
%\textit{loop dead optimization}, este que tem por objetivo remover do corpo 
%do \textit{loop}, códigos que não necessitam estar no corpo do \textit{loop}. 
%Dentre os códigos removidos, é analisado se o mesmo deve ser executado antes ou 
%depois do \textit{loop}, assim realizando uma reordenação das declarações e
%gerando um código mais eficiente.
