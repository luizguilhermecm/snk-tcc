% vim: set ai tw=80 fileencoding=utf8: 
%-------------------------------------------------------------------------------

\chapter{Trabalhos Relacionados}

Nesta seção serão aprensentadas abordagens realizadas por outros autores que 
desenvolveram estudo de técnicas relacionadas ao conteúdo deste trabalho.

Em \textit{"Patterns of optimized loops"} \cite{Tasharofi:2010} os autores
investigam os conhecidos padrões de paralelização de \textit{loops} com
características específicas.  
Assim como neste trabalho os autores procuram identificar em um 
\textit{loop} determinadas características que os ajudem a identificar quais
técnicas são passíveis de serem aplicadas naquele \textit{loop}.

Os autores abordam três padrões: \textit{partitioned loop pattern}, \textit{tiled
loop pattern} e \textit{wavefront pattern}. 
Todos estes padrões utilizam algumas das técnicas abordadas neste trabalho. 
O padrão \textit{partitioned loop pattern} utiliza como solução as técnicas
\textit{loop index set splitting, loop peeling}. 
O padrão \textit{tiled loop pattern} utiliza a técnica \textit{loop unrolling}. 
O padrão \textit{wavefront pattern} utiliza a técinca \textit{loop interchange}.

Não será utilizado nomenclaturas de padrões de \textit{loops} paralelizáveis
neste trabalho, embora, identificar características que ajudem a aplicar a
melhor técnica de tranformação de \textit{loops} será primordial para obter bons
resultados.


Em \textit{"Function inlining and loop unrolling for loop acceleration in
reconfigurable processors"} \cite{Miniskar:2012}, os autores estudam os impactos
do uso de funções \textit{inline} e da técnica \textit{loop unrolling} na
melhora de desempenho de \textit{loops} para uma arquitetura específica.
Embora a arquitetura alvo dos autores não se pareça com a utilizada neste
trabalho, os impactos de \textit{loop unrolling}, tal qual, a quantidade de
vezes que poderá ser aplicada, para a obtenção do melhor resultado este será
um dos pontos que mais exigirão testes e experimentos neste trabalho.

Técnicas de tranformação de \textit{loops} apresentam comportamentos diferentes 
para cada arquitetura. 
Por apresentarem este comportamento muitas vezes empirista, a experiência obtida
pelos autores será de suma importância, para que no final deste trabalho tenhamos
encontrado a melhor forma de utilizar esta técnica.

Embora o uso de funções \textit{inline} apresentarem bons resultados, estas não
serão utilizadas neste trabalho, uma vez que não são técnicas de transformação
de \textit{loops}.

%-------------------------------------------------------------------------------
% Exploitation of parallelism to nested loops with dependence cycles
%-------------------------------------------------------------------------------

%Em \textit{"Exploitation of parallelism to nested loops with dependence
%cyles"} \cite{Chang:2004}, os autores estudam a exploração de paralelismo de
%\textit{loops} contendo ciclos de dependências, com foco na quebra das relações 
%de dependências para então extrair paralelismo destes \textit{loops} com menos
%ou nenhuma relação de dependência.
%
%Os autores começam realizando uma revisão teórica, apresentando os principais
%conceitos de dependência, assim como as principais relações de dependências, 
%\textit{true-dependence}, \textit{output-dependence} e \textit{anti-dependence}.  
%São também apresentados alguns conceitos como \textit{loop independent dependence}, 
%\textit{loop-carried dependence}, \textit{dependence distance vector}, 
%\textit{dependence distance matrix}, \textit{inter-statement} ou
%\textit{intra-statement} dependences e \textit{direction vector matrix} de
%\textit{loops} aninhados.
%
%De acordo com os autores os ciclos de relações \textit{anti-dependence} e
%\textit{true-dependence} podem ser quebradas embora \textit{true-dependence} é 
%inquebrável.
%
%A estratégia para quebrar a relação de dependência \textit{anti-dependence}
%apresentada se baseia no uso de uma técnica de renomeação de variáveis
%chamada \textit{sink}. 
%A partir da qual é criada uma variável nova que será utilizada para manter os
%valores daquela envolvida na relação \textit{anti-dependence}.  
%Para \textit{output-dependence} o mesmo algoritmo é apresentado.
%
%Visto que os que os ciclos de relações de \textit{true-dependence} não pode ser
%quebrados, os autores apresentam várias técnicas que permitem amenizar as
%relações de dependências, o resultado deste último algoritmo é especialmente
%voltado para a aplicação de vetorização.
%
%Os autores mostram que a aplicabilidade de seus algoritmos satisfizeram e
%mantiveram todas as relações de dependências iniciais, assim não alterando o 
%resultado final.
%
%Os resultados obtidos utilizando \textit{vector loops} mostraram que as técnicas
%propostas são muito significativa em termos de \textit{speed-up}, a execução dos 
%programas tranformados foram de 41 a 69 vezes mais rápidos em comparação a execução
%dos programas originais.





%Em \cite{Behera:2006} os autores apresentam uma melhoria no algoritmo 
%\textit{loop dead optimization}, este que tem por objetivo remover do corpo 
%do \textit{loop}, códigos que não necessitam estar no corpo do \textit{loop}. 
%Dentre os códigos removidos, é analisado se o mesmo deve ser executado antes ou 
%depois do \textit{loop}, assim realizando uma reordenação das declarações e
%gerando um código mais eficiente.
