% vim: set ai tw=80 fileencoding=utf8: 
%-------------------------------------------------------------------------------
\chapter*{Introdução}

Diante da  dificuldade de se melhorar processadores \textit{single-core} devido ao alto consumo de energia e as altas
temperaturas, a indústria de processadores adotou como solução à esses
problemas, o desenvolvimento de processadores \textit{multi-core},
além disso também considerou-se o poder de processamento quando se utilizam vários processadores \textit{single-core} 
trabalhando simultaneamente, os quais oferecem grande desempenho, maior eficiência energética e menor custo.
Porém as aplicações desenvolvidas para a arquitetura \textit{single-core} não utilizam-se do poder computacional dos processadores \textit{multi-core}.
Para que as aplicações possam usufruir do alto desempenho que estes processadores oferecem, as aplicações devem ser 
divididas em múltiplas partes para que possam ser executadas em paralelo.

A paralelização de código permite ao programador resolver problemas com maior eficiência, porém projetar e codificar um 
programa paralelo continua sendo uma tarefa difícil, uma vez que além de dividi-lo em pequenas tarefas, deve-se considerar a 
concorrência entre elas, as dependências de dados, entre outros fatores que dificultam o trabalho de paralelização 

