% vim: set ai tw=80 fileencoding=utf8: 
%-------------------------------------------------------------------------------
\chapter{Introdução}

Diante da  dificuldade de se melhorar processadores \textit{single-core} devido 
ao alto consumo de energia e as altas temperaturas, a indústria de processadores 
adotou como solução à esses problemas, o desenvolvimento de processadores 
\textit{multi-core}, além disso também considerou-se o poder de processamento 
quando se utilizam vários processadores \textit{single-core} trabalhando 
simultaneamente, os quais oferecem grande desempenho, maior eficiência energética 
e menor custo.
Porém as aplicações desenvolvidas para a arquitetura \textit{single-core} não 
utilizam-se do poder computacional dos processadores \textit{multi-core}.
Para que as aplicações possam usufruir do alto desempenho que estes processadores 
oferecem, as aplicações devem ser divididas em múltiplas partes para que possam 
ser executadas em paralelo.

A paralelização de código permite ao programador resolver problemas com maior 
eficiência, porém projetar e codificar um programa paralelo continua sendo uma 
tarefa difícil, uma vez que além de dividi-lo em pequenas tarefas, deve-se 
considerar a concorrência entre elas, as dependências de dados, entre outros 
fatores que dificultam o trabalho de paralelização 

Programas computacionalmente custosos, no geral gastam a maior parte do esforço 
na execução de \textit{loops}, assim, a maior parte do ganho de eficiência será
obtido otimizando as partes do software onde há maior esforço computacional.
As técnicas de reestruturação de \textit{loops} como \textit{loop 
fusion, loop fission, loop peeling}, entre outras podem ser utilizadas para
otimizar \textit{loops} e obter ganho de eficiência, melhorar o \textit{data
locality} e até o consumo energético pode ser diminuído \cite{Liu:2004}.

Neste contexto o objetivo deste trabalho será obter ganho de eficiência atravez da 
utilização de técnicas de reestruturação de \textit{loops} e de dependências de 
dados aplicados a um software de equalização de áudio.

O restante deste trabalho está dividido da seguinte forma: O capítulo 2,
  apresenta a fundamentação teória de arquiteturas paralelas, dependência de
  dados e as técnicas de transformações de \textit{loops}; O capítulo 3, são
  apresentados os resultados experimentais; O capítulo 4, a conclusão sobre o
  trablaho.
