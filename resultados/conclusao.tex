% vim: set ai tw=80 fileencoding=utf8: 
%-------------------------------------------------------------------------------
\chapter{Conclusão}

Neste trabalho foi discutido um procedimento de paralelização de tarefas sobre 
um algoritmo pré-existente. 
Inicialmente apresentou-se uma fundamentação teórica com técnicas clássicas de 
paralelização, tanto em \textit{hardware} como em \textit{software}. 
Em seguida, foi descrita com maior profundidade a programação envolvendo 
paralelismo, destacando-se as técnicas transformação de loops.

O objetivo desse trabalho é aplicar técnicas de transformação de \textit{loops} 
em uma aplicação para melhora de seu desempenho. 
A aplicação utilizada foi o programa de equalização de áudio \textit{wat}. 
Embora existam muitas técnicas, a obtenção de benefícios em usá-las é uma 
característica empírica. 
Os resultados obtidos com as otimizações mostraram que o uso de técnicas de
transformação de \textit{loops} podem significar um grande impacto no desempenho
do programa.
Casos onde as técnicas não obtiveram resultados significativos, acabou por ficar
evidente que o custo computacional estava em chamadas de função que não podem
ser melhoradas com transformação de \textit{loop}.

Dessa forma, pode-se elencar alguns trabalhos futuros a fim de se chegar em 
resultados ainda mais expressivos do que os aqui obtidos. 

Entre os trabalhos futuros é desejável que se aborde toda a questão de 
transformação de \textit{loops} voltados a uma arquitetura específica. 
Esse tipo de investigação pode ser motivada por possíveis comportamentos 
diferenciados de um mesmo algoritmo conforme a plataforma de 
\textit{hardware} tal como memória \textit{cache} e \textit{intruction cache}.

Técnicas de otimização de código a serem aplicados naqueles que estão no
corpo do \textit{loop} pode ser primordial para que o custo compucional do
\textit{loop} seja reduzido, tal como utilizar funções \textit{inline} e
substituição de variáveis, evitando gasto com acesso a memória através de
ponteiros.

A utilização das ténicas de transformações de \textit{loops} voltadas à
utilização de \textit{threads} em processadores \textit{multi-cores}.
