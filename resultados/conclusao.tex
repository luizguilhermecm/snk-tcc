% vim: set ai tw=80 fileencoding=utf8: 
%-------------------------------------------------------------------------------
\chapter{Conclusão}

Neste trabalho foi discutido um procedimento de paralelização de tarefas sobre 
um algoritmo pré-existente. 
Inicialmente apresentou-se uma fundamentação teórica com técnicas clássicas de 
paralelização, tanto em \textit{hardware} como em \textit{software}. 
Em seguida, foi descrita com maior profundidade a programação envolvendo 
paralelismo, destacando-se as técnicas transformação de loops.

O objetivo desse trabalho é aplicar técnicas de transformação de \textit{loops} 
em uma aplicação para melhora de seu desempenho. 
A aplicação utilizada foi o programa de equalização de áudio \textit{wat}. 
Embora existam muitas técnicas, a obtenção de benefícios em usá-las é uma 
característica empírica. 
Devido a isso, os resultados obtidos após as alterações, 
mesmo que apresentando maior desempenho, não demonstram vantagem significativa.

Dessa forma, pode-se elencar alguns trabalhos futuros a fim de se chegar em 
resultados mais expressivos do que os aqui obtidos. 
Entre os trabalhos futuros é desejável que se aborde toda a questão de 
transformação de \textit{loops} voltados a uma arquitetura específica. 
Esse tipo de investigação pode ser motivada por possíveis comportamentos 
diferenciados de um mesmo algoritmo conforme a plataforma de 
\textit{hardware}.
