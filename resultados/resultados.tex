% vim: set ai tw=80 sw=2 fileencoding=utf8: 
%-------------------------------------------------------------------------------
\chapter{Experimentos e Resultados}

Este capítulo apresenta os resultado obtidos na aplicação das técnicas de
transformação de \textit{loops} e também
os experimentos realizados.

Para a realização dos experimentos foi escolhido o programa
\textit{wat\footnote{https://github.com/luizguilhermecm/wat}}, que é um
equalizador de áudio de 1 e 2 canais.

%A escolha de \textit{wat} foi motivado pela fato deste trabalhar com árquivos de
%áudio, ou seja, grandes vetores de dados e \textit{loops} que iterem sobre estes
%dados. 
%Estes \textit{loops} oferecem uma grande chance de ganho de desempenho por
%iterarem um grande número de vezes.
%Ganho de desempenho sobre \textit{wat} resultaria em uma equalização mais rápida.

Á escolha do \textit{wat} foi motivada pelo fato de ser compatível com arquivos 
compostos de grandes vetores de dados. 
Arquivos de áudio apresentam essa composição. 
Sobre esses vetores são iterados \textit{loops} uma grande quantidade de vezes. 
Isso permite que se ganhe desempenho aplicando modificações sobre os loops do 
\textit{wat}. Tal ganho de desempenho resulta em uma equalização mais rápida.

%-------------------------------------------------------------------------------
\section{Experimentos}

Nesta seção serão apresentados os testes e experimentos realizados sobre os
\textit{loops} extraídos de \textit{wat}, assim como seus resultados
individuais. 

\subsection{Experimento 1}
Analisando o \textit{loop} do algoritmo~\ref{read_data_orig} o qual 
transforma dois \textit{bytes} em um \textit{double}.

\begin{algorithm}[H]
  \caption{\textit{Loop} extraído do \textit{wat}.}
    \label{read_data_orig}
\lstinputlisting[language=c]{resultados/src/read_data_orig.c}
\end{algorithm}


Devido a existência de uma declaração \textit{if} no corpo do \textit{loop} e
sendo este \textit{if} independente do \textit{loop}. 
Foi aplicado a técnica \textit{loop unswitching}, que retira a comparação
\textit{if} do corpo do \textit{loop} envolvendo um novo \textit{loop} para cada caso
da comparação, assim nenhum dos \textit{loops} terão declações de comparação em seu
corpo.
O algoritmo~\ref{read_data_opt} mostra o resultado do
algoritmo~\ref{read_data_orig} após a alteração com o uso de \textit{loop unswitching}.

Considerando o algoritmo~\ref{read_data_opt} será aplicado a técnica 
\textit{loop unroll} com o intuito de diminuir o número de iterações e 
consequentemente a
quantidade de comparações realizadas no \textit{loop} em duas vezes. 
O algoritmo~\ref{read_data_unroll} mostra o resultado do \textit{loop
unroll} duplicando o corpo do \textit{loop}.

Considerando o algoritmo~\ref{read_data_opt} onde o primeiro \textit{loop} itera
sobre dois vetores e o segundo \textit{loop} itera
sobre três vetores, será aplicado \textit{loop fission} dividindo-os de forma
que cada \textit{loop} itere somente sobre um vetor.
O algoritmo~\ref{read_data_fission} mostra o resultado de \textit{loop fission}
no algoritmo~\ref{read_data_opt}.

O algoritmo~\ref{read_data_fission} apresenta cinco \textit{loops}, cada um
iterando sobre um vetor. Será aplicado \textit{loop unrolling} dobrando o corpo
de cada um dos \textit{loops} para que o número de iterações seja dividido por
dois.
O algoritmo~\ref{read_data_unrofis} mostra o resultado de \textit{loop
unrolling} no algoritmo~\ref{read_data_fission}.

A tabela~\ref{tabela_read_data} apresenta os resultados obtidos sobre o
algoritmo~\ref{read_data_orig}.

\begin{table}[H]
  \caption{Resultados das técnicas sobre o algoritmo~\ref{read_data_orig}.}
  \label{tabela_read_data}
\begin{center}
  \begin{tabular}{c|c|c}
    Algoritmo & Técnicas aplicadas & Tempo de execução\\
    \hline
    Algoritmo~\ref{read_data_orig} & nenhuma & 111 \textit{ms} \\
    \hline
    Algoritmo~\ref{read_data_opt} & \textit{loop unswitching} & 111 \textit{ms} \\
    \hline
    Algoritmo~\ref{read_data_unroll} & \textit{loop unswitching, loop unroll} & 221 \textit{ms} \\
    \hline
    Algoritmo~\ref{read_data_fission} & \textit{loop unswitching, loop fission} & 111 \textit{ms} \\
    \hline
  Algoritmo~\ref{read_data_unrofis} & \textit{loop unswitching, loop fisson, loop unroll} & 111 \textit{ms} \\
    \hline
  \end{tabular}
\end{center}
\end{table}


\begin{algorithm}[H]
  \caption{\textit{Loop unswitching} no algoritmo~\ref{read_data_orig}.}
    \label{read_data_opt}
\lstinputlisting[language=c]{resultados/src/read_data_opt.c}
\end{algorithm}

\begin{algorithm}[H]
  \caption{\textit{Loop unroll} no algoritmo~\ref{read_data_opt}.}
    \label{read_data_unroll}
\lstinputlisting[language=c]{resultados/src/read_data_unroll.c}
\end{algorithm}

\begin{algorithm}[H]
  \caption{\textit{Loop fission} no algoritmo~\ref{read_data_opt}.}
    \label{read_data_fission}
\lstinputlisting[language=c]{resultados/src/read_data_fission.c}
\end{algorithm}

\begin{algorithm}[H]
  \caption{\textit{Loop unrolling} no algoritmo~\ref{read_data_fission}.}
    \label{read_data_unrofis}
\lstinputlisting[language=c, basicstyle=\tiny ]{resultados/src/read_data_unrofis.c}
\end{algorithm}

%-------------------------------------------------------------------------------
\subsection{Experimento 2}

Analisando o \textit{loop} do algoritmo~\ref{fix_data_orig} que rearranja os dados
para o calculo da FFT. 

\begin{algorithm}[H]
  \caption{Loop extraído do \textit{wat}.}
    \label{fix_data_orig}
\lstinputlisting[language=c]{resultados/src/fix_data_orig.c}
\end{algorithm}

O algoritmo~\ref{fix_data_orig} é parecido com o algoritmo~\ref{read_data_orig}, uma
vez que ambos contêm uma declaração \textit{if} independente do \textit{loop} em seu corpo. 
Assim, também foi aplicado a técnica \textit{loop unswitching} no
algoritmo~\ref{fix_data_orig}.
O algoritmo~\ref{fix_data_opt} mostra o resultado da aplicação de \textit{loop
unswitching} no algoritmo~\ref{fix_data_orig}.

Com o intuito de diminuir o número de iterações será aplicado \textit{loop
unroll} no algoritmo~\ref{fix_data_opt}. 
O algoritmo~\ref{fix_data_unroll} mostra o resultado de \textit{loop unroll} no
algoritmo~\ref{fix_data_opt}.

Considerando o algoritmo~\ref{fix_data_opt} onde o segundo \textit{loop} itera
sobre dois vetores. Será aplicado \textit{loop fission}, dividindo-o em dois 
\textit{loops} onde cada um itere em um vetor.
O algoritmo~\ref{fix_data_fission} mostra o resultado de \textit{loop fission}
no algoritmo~\ref{fix_data_opt}.

Com algoritmo~\ref{fix_data_fission} agora com três \textit{loops}. Será
aplicado \textit{loop unrolling} neste para que o número de iterações em cada
\textit{loop} seja dividido por dois. 
O algoritmo~\ref{fix_data_unrofis} mostra o resultado de \textit{loop unrolling}
no algoritmo~\ref{fix_data_fission}.

A tabela~\ref{tabela_fix_data} apresenta os resultados obtidos sobre o
algoritmo~\ref{fix_data_orig}.

\begin{table}[H]
  \caption{Resultados das técnicas sobre o algoritmo~\ref{fix_data_orig}.}
  \label{tabela_fix_data}
\begin{center}
  \begin{tabular}{c|c|c}
    Algoritmo & Técnicas aplicadas & Tempo de execução\\
    \hline
    Algoritmo~\ref{fix_data_orig} & nenhuma & 111 \textit{ms} \\
    \hline
    Algoritmo~\ref{fix_data_opt} & \textit{loop unswitching} & 111 \textit{ms} \\
    \hline
    Algoritmo~\ref{fix_data_unroll} & \textit{loop unswitching, loop unroll} & 221 \textit{ms} \\
    \hline
    Algoritmo~\ref{fix_data_fission} & \textit{loop unswitching, loop fission} & 111 \textit{ms} \\
    \hline
  Algoritmo~\ref{fix_data_unrofis} & \textit{loop unswitching, loop fisson, loop unroll} & 111 \textit{ms} \\
    \hline
  \end{tabular}
\end{center}
\end{table}



\begin{algorithm}[H]
  \caption{\textit{Loop unswitching} no algoritmo~\ref{fix_data_orig}.}
    \label{fix_data_opt}
\lstinputlisting[language=c]{resultados/src/fix_data_opt.c}
\end{algorithm}

\begin{algorithm}[H]
  \caption{\textit{Loop unroll} no algoritmo~\ref{fix_data_opt}.}
    \label{fix_data_unroll}
\lstinputlisting[language=c]{resultados/src/fix_data_unroll.c}
\end{algorithm}

\begin{algorithm}[H]
  \caption{\textit{Loop fission} no algoritmo~\ref{fix_data_opt}.}
    \label{fix_data_fission}
\lstinputlisting[language=c]{resultados/src/fix_data_fission.c}
\end{algorithm}

\begin{algorithm}[H]
  \caption{\textit{Loop unrolling} no algoritmo~\ref{fix_data_fission}.}
    \label{fix_data_unrofis}
\lstinputlisting[language=c]{resultados/src/fix_data_unrofis.c}
\end{algorithm}

%%-------------------------------------------------------------------------------
\subsection{Experimento 3}

Analisando o \textit{loop} do algoritmo~\ref{equalize_orig} o qual multiplica um dado 
vetor por onze fatores de equalização.

\begin{algorithm}[H]
  \caption{\textit{Loop} extraído do \textit{wat}.}
\label{equalize_orig}
\lstinputlisting[language=c]{resultados/src/equalize_orig.c}
\end{algorithm}

O algoritmo~\ref{equalize_orig} também possui declaração de comparação no corpo do
\textit{loop}, porém as comparações do algoritmo~\ref{equalize_orig} dependem da
iteração do \textit{loop}, sendo assim não é permitido a aplicação de
\textit{loop unswitching}.

Para este \textit{loop} foi aplicado a técnica \textit{index set
splitting}, uma vez que os limites do \textit{loop} e também o valor de cada
comparação realizada ser fixo no programa, o \textit{index set splitting} irá
dividir o \textit{loop} em onze \textit{loops} com o espaço de iteração variando
de \textit{BAND\_MIN} até \textit{BAND\_0}, \textit{BAND\_0} até \textit{BAND\_1} e
assim até o ultimo \textit{loop} que irá iterar de \textit{BAND\_9} até
\textit{BAND\_MAX}. Com isso será eliminado todas as declarações de comparação e
continuará iterando o mesmo número de vezes.

O algoritmo~\ref{equalize_opt} mostra o resultado da aplicação de 
\textit{index set splitting} no algoritmo~\ref{equalize_orig}.

Com o intuito de diminuir o número de iterações no algoritmo~\ref{equalize_opt}
foi aplicado \textit{loop unrolling} em cada um dos onze \textit{loops}. 
O algoritmo~\ref{equalize_unroll} mostra o resultado de \textit{loop unrolling}
no algoritmo~\ref{equalize_opt}.

A tabela~\ref{tabela_equalize} apresenta os resultados obtidos sobre o
algoritmo~\ref{equalize_orig}.

\begin{table}[H]
  \caption{Resultados das técnicas sobre o algoritmo~\ref{equalize_orig}.}
  \label{tabela_equalize}
\begin{center}
  \begin{tabular}{c|c|c}
    Algoritmo & Técnicas aplicadas & Tempo de execução\\
    \hline
    Algoritmo~\ref{equalize_orig} & nenhuma & 111 \textit{ms} \\
    \hline
    Algoritmo~\ref{equalize_opt} & \textit{index set splitting} & 111 \textit{ms} \\
    \hline
    Algoritmo~\ref{equalize_unroll} & \textit{index set splitting, loop unroll} & 221 \textit{ms} \\
    \hline
  \end{tabular}
\end{center}
\end{table}



\begin{algorithm}[H]
  \caption{\textit{Index set splitting} no algoritmo~\ref{equalize_orig}.}
\label{equalize_opt}
\lstinputlisting[language=c]{resultados/src/equalize_opt.c}
\end{algorithm}

\begin{algorithm}[H]
  \caption{\textit{Loop unrolling} no algoritmo~\ref{equalize_opt}.}
\label{equalize_unroll}
\lstinputlisting[language=c, basicstyle=\tiny]{resultados/src/equalize_unroll.c}
\end{algorithm}

%%-------------------------------------------------------------------------------
\subsection{Experimento 4}

Analisando o \textit{loop} do algoritmo~\ref{back_data_orig} que transforma os 
dados após a aplicação da FFT.

\begin{algorithm}[H]
  \caption{\textit{Loop} extraído do \textit{wat}.}
\label{back_data_orig}
\lstinputlisting[language=c]{resultados/src/back_data_orig.c}
\end{algorithm}

O algoritmo~\ref{back_data_orig} também contém uma declaração \textit{if} no corpo do
\textit{loop} que é independente do \textit{loop}. Assim, foi também
aplicado neste \textit{loop} a técnica \textit{loop unswitching}.
O algoritmo~\ref{back_data_opt} mostra o resultado de \textit{loop unswitching} no
algoritmo~\ref{back_data_orig}.

Para que o número de iteração no algoritmo~\ref{back_data_opt} diminua foi
aplicado \textit{loop unrolling}. O algoritmo~\ref{back_data_unroll} mostra o
resultado de \textit{loop unrolling} no algoritmo~\ref{back_data_orig}.

O segundo \textit{loop} do algoritmo~\ref{back_data_opt} está iterando sobre dois
vetores do lado direito das declarações de atribuição, sendo assim, foi
aplicado \textit{loop fission}, dividindo-os em dois \textit{loops}.
O algoritmo~\ref{back_data_fission} mostra o resultado de \textit{loop fission}
no algoritmo~\ref{back_data_opt}.

Para diminuir o número de iterações dos \textit{loops} do
algoritmo~\ref{back_data_fission} foi aplicado \textit{loop unrolling} em cada um
dos \textit{loops}. O algoritmo~\ref{back_data_unrofis} mostra o resultado de
\textit{loop unrolling} no algoritmo~\ref{back_data_fission}.


A tabela~\ref{tabela_back_data} apresenta os resultados obtidos sobre o
algoritmo~\ref{back_data_orig}.

\begin{table}[H]
  \caption{Resultados das técnicas sobre o algoritmo~\ref{back_data_orig}.}
  \label{tabela_back_data}
\begin{center}
  \begin{tabular}{c|c|c}
    Algoritmo & Técnicas aplicadas & Tempo de execução\\
    \hline
    Algoritmo~\ref{back_data_orig} & nenhuma & 111 \textit{ms} \\
    \hline
    Algoritmo~\ref{back_data_opt} & \textit{loop unswitching} & 111 \textit{ms} \\
    \hline
    Algoritmo~\ref{back_data_unroll} & \textit{loop unswitching, loop unroll} & 221 \textit{ms} \\
    \hline
    Algoritmo~\ref{back_data_fission} & \textit{loop unswitching, loop fission} & 111 \textit{ms} \\
    \hline
  Algoritmo~\ref{back_data_unrofis} & \textit{loop unswitching, loop fisson, loop unroll} & 111 \textit{ms} \\
    \hline
  \end{tabular}
\end{center}
\end{table}



\begin{algorithm}[H]
\caption{\textit{Loop unswitching} no algoritmo~\ref{back_data_orig}.}
\label{back_data_opt}
\lstinputlisting[language=c]{resultados/src/back_data_opt.c}
\end{algorithm}


\begin{algorithm}[H]
\caption{\textit{Loop unrolling} no algoritmo~\ref{back_data_opt}.}
\label{back_data_unroll}
\lstinputlisting[language=c]{resultados/src/back_data_unroll.c}
\end{algorithm}

\begin{algorithm}[H]
\caption{\textit{Loop fission} no algoritmo~\ref{back_data_opt}.}
\label{back_data_fission}
\lstinputlisting[language=c]{resultados/src/back_data_fission.c}
\end{algorithm}

\begin{algorithm}[H]
\caption{\textit{Loop unrolling} no algoritmo~\ref{back_data_fission}.}
\label{back_data_unrofis}
\lstinputlisting[language=c]{resultados/src/back_data_unrofis.c}
\end{algorithm}
%%-------------------------------------------------------------------------------
\subsection{Experimento 5}

Analisando o algoritmo~\ref{convert_double_orig} que apresenta 
dois \textit{loops},
sendo que o primeiro converte um vetor \textit{double} em um vetor 
\textit{short int} e o segundo \textit{loop} combina os \textit{bytes} do vetor
\textit{short int} em um vetor \textit{unsigned char}.

\begin{algorithm}[H]
  \caption{\textit{Loop} extraído do \textit{wat}.}
\label{convert_double_orig}
\lstinputlisting[language=c]{resultados/src/convert_double_orig.c}
\end{algorithm}

O algoritmo~\ref{convert_double_orig} contém uma declaração \textit{if} no corpo de
cada um dos \textit{loops}, onde ambos são independentes de seu respectivo \textit{loop}. 
Sendo assim, foi aplicada a técnica \textit{loop unswitching} em cada um dos
\textit{loops} do algoritmo~\ref{convert_double_orig} e resultando no
algoritmo~\ref{convert_double_opt}.

O algoritmo~\ref{convert_double_opt} agora sem declaração \textit{if} no corpo
do \textit{loops} apresenta dois \textit{loops} para cada resultado de
\textit{if} ou \textit{else if}, sendo assim, foi feito uma reordenação das
declarações com o intuito de melhorar a legibilidade do código para aplicação de
outras técnicas.
O algoritmo~\ref{convert_double_reo} mostra o resultado da reordenação das
declarações do algoritmo~\ref{convert_double_opt}, onde cada resultado das
comparações possuem dois \textit{loops}, a ordem de computação dos
\textit{loops} foi mantida para que o significado do programa não se altere.

Analisando o algoritmo~\ref{convert_double_reo}, nota-se que os 
\textit{loops} tem como
espaço de iteração o valor de \textit{wi->nb\_samples}, embora o segundo
\textit{loop} itere sobre \textit{wi->nb\_samples * iterator},  é possível aplicar
a técnica \textit{loop fusion} nos \textit{loops}, assim unindo-os em um
único \textit{loop}, sendo necessário ajustar o espaço de iteração. 
Devido ao
segundo \textit{loop} utilizar os dados computados pelo primeiro \textit{loop},
será necessário garantir que essa dependência seja mantida.

A aplicação do \textit{loop fusion} sobre o
algoritmo~\ref{convert_double_opt} pode se tornar confusa devido a aplicação do
\textit{loop unswitching}, que torna mais complexa a leitura do código. 
A reordenação de declarações fez com que o algoritmo~\ref{convert_double_opt}
ficasse mais legível.
O algoritmo~\ref{convert_double_fusion} mostra o 
resultado de \textit{loop fusion} no algoritmo~\ref{convert_double_reo}. 

O algoritmo~\ref{convert_double_fusion} apresenta dois \textit{loops}, onde o
segundo \textit{loop} itera sobre uma cinco vetores com um número de iteração
igual ao número de \textit{bytes} do arquivo de áudio. 
Será aplicado \textit{loop fission} no segundo \textit{loop} com o intuito de
dividi-lo em dois \textit{loops} onde cada um itere sobre um dos canais de
áudio.
O algoritmo~\ref{convert_double_fission} mostra o resultado de \textit{loop
fission} no algoritmo~\ref{convert_double_fusion}.


A tabela~\ref{tabela_convert_double} apresenta os resultados obtidos sobre o
algoritmo~\ref{convert_double_orig}.


\begin{table}[H]
  \caption{Resultados das técnicas sobre o algoritmo~\ref{convert_double_orig}.}
  \label{tabela_convert_double}
\begin{center}
  \begin{tabular}{c|c|c}
    Algoritmo & Técnicas aplicadas & Tempo de execução\\
    \hline
    Algoritmo~\ref{convert_double_orig} & nenhuma & 111 \textit{ms} \\
    \hline
    Algoritmo~\ref{convert_double_opt} & \textit{loop unswitching} & 111 \textit{ms} \\
    \hline
    Algoritmo~\ref{convert_double_fusion} & \textit{loop unswitching, loop fusion} & 221 \textit{ms} \\
    \hline
    Algoritmo~\ref{convert_double_fission} & \textit{loop unswitching, loop fusion, loop fission} & 111 \textit{ms} \\
    \hline
  \end{tabular}
\end{center}
\end{table}



\begin{algorithm}[H]
  \caption{\textit{Loop unswitching} no algoritmo~\ref{convert_double_orig}.}
\label{convert_double_opt}
\lstinputlisting[language=c]{resultados/src/convert_double_opt.c}
\end{algorithm}

\begin{algorithm}[H]
  \caption{Reordenação das declarações do algoritmo~\ref{convert_double_opt}.}
\label{convert_double_reo}
\lstinputlisting[language=c]{resultados/src/convert_double_reo.c}
\end{algorithm}

\begin{algorithm}[H]
  \caption{\textit{Loop fusion} no algoritmo~\ref{convert_double_reo}.}
\label{convert_double_fusion}
\lstinputlisting[language=c]{resultados/src/convert_double_fusion.c}
\end{algorithm}

\begin{algorithm}[H]
  \caption{\textit{Loop fission} no algoritmo~\ref{convert_double_fusion}.}
\label{convert_double_fission}
\lstinputlisting[language=c]{resultados/src/convert_double_fission.c}
\end{algorithm}

%%-------------------------------------------------------------------------------
\subsection{Experimento 6}

Analisando o \textit{loop} do algoritmo~\ref{save_file_orig} que escreve em arquivo o 
vetor \textit{buffer}.

\begin{algorithm}[H]
  \caption{\textit{Loop} extraído do \textit{wat}.}
\label{save_file_orig}
\lstinputlisting[language=c]{resultados/src/save_file_orig.c}
\end{algorithm}

Para diminuir o número de iterações no algoritmo~\ref{save_file_orig} foi
aplicado \textit{loop unrolling}. O algoritmo~\ref{save_file_unroll} mostra o
resultado de \textit{loop unrolling} no algoritmo~\ref{save_file_orig}.

\begin{algorithm}[H]
  \caption{\textit{Loop unrolling} no algoritmo~\ref{save_file_orig}.}
\label{save_file_unroll}
\lstinputlisting[language=c]{resultados/src/save_file_unroll.c}
\end{algorithm}

A tabela~\ref{tabela_save_file} apresenta os resultados obtidos sobre o
algoritmo~\ref{save_file_orig}.


\begin{table}[H]
  \caption{Resultados das técnicas sobre o algoritmo~\ref{save_file_orig}.}
  \label{tabela_save_file}
\begin{center}
  \begin{tabular}{c|c|c}
    Algoritmo & Técnicas aplicadas & Tempo de execução\\
    \hline
    Algoritmo~\ref{save_file_orig} & nenhuma & 111 \textit{ms} \\
    \hline
    Algoritmo~\ref{save_file_unroll} & \textit{loop unrolling} & 221 \textit{ms} \\
    \hline
  \end{tabular}
\end{center}
\end{table}



%\begin{algorithm}
%\caption{Loop com declaração if }
%\label{read_data_2}
%\begin{lstlisting}[language=c]
%
%\end{lstlisting}
%\end{algorithm}

%%-------------------------------------------------------------------------------
\section{Resultados Experimentais}

Analisando todos os \textit{loops} otimizados houve uma melhora na desempenho
destes em um total de XXX \textit{ms}. Considerando o tempo total do programa a
melhora do desempenho passa a não ser tão considerável devido ao fato de o
cálculo da FFT já estar otimizado e nada mais pode ser feito para melhora-lo.

%\lstinputlisting[language=c]{resultados/src/file.c}
