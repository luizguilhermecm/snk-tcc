% vim: set ai tw=80 sw=2 fileencoding=utf8: 
%-------------------------------------------------------------------------------

\chapter{Resultados}

\section{Teste}
Analisando o \textit{loop} do algoritmo~\ref{read_data_1} que 
transforma dois \textit{bytes} em um \textit{double}.

Devido a existência de uma declaração \textit{if} no corpo do \textit{loop} e
sendo este \textit{if} independente do \textit{loop} foi
aplicado a técnica \textit{loop unswitching}, que retira a comparação
\textit{if} do corpo do \textit{loop} envolvendo um \textit{loop} para cada caso
da comparação, assim nenhum dos \textit{loops} terão declações de comparação em seu
corpo.

\begin{algorithm}
    \caption{Declaração \textit{if} independente do loop}
    \label{read_data_1}
    \begin{lstlisting}[language=c]
    it = i = 0;
    while(it < wi->wav_header->subchunk2_size){
        wi->left_side[i] = bytes_to_double(buffer[it],
            buffer[it + 1]);
        wi->zero_data[i] = 0;
        it += 2;
        if(wi->wav_header->num_channels == 2){
            wi->right_side[i] = bytes_to_double(buffer[it],
                buffer[it + 1]);
            it+=2;
        }
        i++;
    }
    \end{lstlisting}
\end{algorithm}

O algoritmo~\ref{read_data_2} mostra o resultado do algoritmo~\ref{read_data_1} 
após o \textit{loop unswitching}.

Com o uso de \textit{loop unswitching} o tempo médio gasto na computação deste
\textit{loop}, passou de 696 ms para 461 ms, este ganho deve-se ao fato de não
ser mais necessário a computação do \textit{if} a cada iteração. 


\begin{algorithm}
    \caption{Loop sem declaração if }
    \label{read_data_2}
    \begin{lstlisting}[language=c]
    it = i = 0;
    if(wi->wav_header->num_channels == 1){
        while(it < wi->wav_header->subchunk2_size){
            wi->left_side[i] = bytes_to_double(buffer[it], 
                buffer[it + 1]);
            wi->zero_data[i] = 0;
            it += 2;
            i++;
        }
    }
    else if(wi->wav_header->num_channels == 2){
        while(it < wi->wav_header->subchunk2_size){
            wi->left_side[i] = bytes_to_double(buffer[it], 
                buffer[it + 1]);
            wi->zero_data[i] = 0;
            it += 2;
            wi->right_side[i] = bytes_to_double(buffer[it], 
                buffer[it + 1]);
            it+=2;
            i++;
        }
    }
    \end{lstlisting}
\end{algorithm}

%-------------------------------------------------------------------------------

Analisando o \textit{loop} do algoritmo~\ref{fix_data_1} que rearranja os dados
para computação da FFT. 

O algoritmo~\ref{fix_data_1} é parecido com o algoritmo~\ref{read_data_1}, uma
vez que ambos contêm declaração \textit{if} independente do \textit{loop} em seu corpo. 
Assim, também foi aplicado a técnica \textit{loop unswitching} no
algoritmo~\ref{fix_data_1}.


\begin{algorithm}
    \caption{Loop com declaração if }
    \label{fix_data_1}
    \begin{lstlisting}[language=c]
    for(i = 0; i < wi->nb_samples; i++){
        wi->left_fixed[2*i+1] = wi->left_side[i];
        wi->left_fixed[2*i+2] = 0;
        if(wi->wav_header->num_channels == 2){
            wi->right_fixed[2*i+1] = wi->right_side[i];
            wi->right_fixed[2*i+2] = 0;
        }
    }
    \end{lstlisting}
\end{algorithm}

O algoritmo~\ref{fix_data_2} mostra o resultado da aplicação de \textit{loop
unswitching} no algoritmo~\ref{fix_data_1}.

\begin{algorithm}
    \caption{Loop sem declaração if }
    \label{fix_data_2}
    \begin{lstlisting}[language=c]
    if(wi->wav_header->num_channels == 1){
        for(i = 0; i < wi->nb_samples; i++){
            wi->left_fixed[2*i+1] = wi->left_side[i];
            wi->left_fixed[2*i+2] = 0;
            wi->right_fixed[2*i+2] = 0;
        }
    }
    if(wi->wav_header->num_channels == 2){
        for(i = 0; i < wi->nb_samples; i++){
            wi->left_fixed[2*i+1] = wi->left_side[i];
            wi->left_fixed[2*i+2] = 0;
            wi->right_fixed[2*i+1] = wi->right_side[i];
            wi->right_fixed[2*i+2] = 0;
        }

    }
    \end{lstlisting}
\end{algorithm}

Com o uso de \textit{loop unswitching} o tempo médio gasto na computação passou
de XXX \textit{ms} para YYY \textit{ms}, este ganho deve ao fato de não ser
realizada a computação da declaração \textit{if} a cada iteração do
\textit{loop}.

%-------------------------------------------------------------------------------

Analisando o algoritmo~\ref{equalize_1} que multiplica um dado vetor por um
11 fatores de equalização.

Assim como os algoritmos~\ref{read_data_1} e~\ref{fix_data_1} o
algoritmo~\ref{equalize_1} também possui declaração de comparação no corpo do
\textit{loop}, porém as comparações do algoritmo~\ref{equalize_1} dependem da
iteração do \textit{loop}, sendo assim não é permitido a aplicação de
\textit{loop unswitching}.

Para otimizar este \textit{loop} será aplicado a técnica \textit{index set
splitting}, uma vez que os limites do \textit{loop} e também o valor de cada
comparação realizada ser fixo no programa, o \textit{index set splitting} irá
dividir o \textit{loop} em 11 \textit{loops} com o espaço de iteração variando
em vez de \textit{BAND\_MIN} até \textit{BAND\_MAX}, variando de
\textit{BAND\_MIN} até \textit{BAND\_0}, \textit{BAND\_0} até \textit{BAND\_1} e
assim até o ultimo \textit{loop} que irá iterar de \textit{BAND\_9} até
\textit{BAND\_MAX}. Com isso será eliminado todas as declarações de comparação e
continuará iterando o mesmo número de vezes.

\begin{algorithm}
\caption{Loop com declaração if }
\label{equalize_1}
\begin{lstlisting}[language=c]
for(i = BAND_MIN; i < BAND_MAX; i++){
    if(i < BAND_0)
        temp[i] *= wi->factors->fac_0;
    else if(i < BAND_1)
        temp[i] *= wi->factors->fac_1;
    else if(i < BAND_2)
        temp[i] *= wi->factors->fac_2;
    else if(i < BAND_3)
        temp[i] *= wi->factors->fac_3;
    else if(i < BAND_4)
        temp[i] *= wi->factors->fac_4;
    else if(i < BAND_5)
        temp[i] *= wi->factors->fac_5;
    else if(i < BAND_6)
        temp[i] *= wi->factors->fac_6;
    else if(i < BAND_7)
        temp[i] *= wi->factors->fac_7;
    else if(i < BAND_8)
        temp[i] *= wi->factors->fac_8;
    else if(i < BAND_9)
        temp[i] *= wi->factors->fac_9;
    else
        temp[i] *= wi->factors->fac_10;
}
\end{lstlisting}
\end{algorithm}

O algoritmo~\ref{equalize_2} mostra o resultado do algoritmo~\ref{equalize_1}
após o utilização da técnica \textit{index set splitting}.

\begin{algorithm}
\caption{Loop sem declarações de comparação}
\label{equalize_2}
\begin{lstlisting}[language=c]
for(i = BAND_MIN; i < BAND_0; i++)
    temp[i] *= wi->factors->fac_0;

for(i = BAND_0; i < BAND_1; i++)
    temp[i] *= wi->factors->fac_1;

for(i = BAND_1; i < BAND_2; i++)
    temp[i] *= wi->factors->fac_2;

for(i = BAND_2; i < BAND_3; i++)
    temp[i] *= wi->factors->fac_3;

for(i = BAND_3; i < BAND_4; i++)
    temp[i] *= wi->factors->fac_4;

for(i = BAND_4; i < BAND_5; i++)
    temp[i] *= wi->factors->fac_5;

for(i = BAND_5; i < BAND_6; i++)
    temp[i] *= wi->factors->fac_6;

for(i = BAND_6; i < BAND_7; i++)
    temp[i] *= wi->factors->fac_7;

for(i = BAND_7; i < BAND_8; i++)
    temp[i] *= wi->factors->fac_8;

for(i = BAND_8; i < BAND_9; i++)
    temp[i] *= wi->factors->fac_9;

for(i = BAND_9; i < BAND_MAX; i++)
    temp[i] *= wi->factors->fac_10;
\end{lstlisting}
\end{algorithm}

Com o uso de \textit{index set splitting} o tempo médio gasto na computação
passou de XXX \textit{ms} para YYY \textit{ms}, esse ganho deve-se ao fato de
que não será mais computado nenhuma comparação e o espaço de iteração continua o
mesmo.

%-------------------------------------------------------------------------------

Analisando o algoritmo~\ref{back_data_1} que transforma os dados após a
aplicação da FFT.

Como já visto anteriormente nos algoritmos~\ref{read_data_1} e~\ref{fix_data_1}
o algoritmo~\ref{back_data_1} contém uma declaração \textit{if} no corpo do
\textit{loop} que também é independente do \textit{loop}. Assim, será também
aplicado neste \textit{loop} a técnica \textit{loop unswitching}.

\begin{algorithm}
\caption{Loop com declaração if }
\label{back_data_1}
\begin{lstlisting}[language=c]
for(i = 0; i < wi->nb_samples; i++){
    wi->left_side[i] = wi->left_fixed[2*i+1]; 
    if(wi->wav_header->num_channels == 2){
        wi->right_side[i] = wi->right_fixed[2*i+1]; 
    }
}
\end{lstlisting}
\end{algorithm}

O algoritmo~\ref{back_data_2} mostra o resultado de \textit{loop unswitching} no
algoritmo~\ref{back_data_1}.

\begin{algorithm}
\caption{Loop sem declaração if }
\label{back_data_2}
\begin{lstlisting}[language=c]
if(wi->wav_header->num_channels == 1){
    for(i = 0; i < wi->nb_samples; i++){
        wi->left_side[i] = wi->left_fixed[2*i+1]; 
    }
}
else if(wi->wav_header->num_channels == 2){
    for(i = 0; i < wi->nb_samples; i++){
        wi->left_side[i] = wi->left_fixed[2*i+1]; 
        wi->right_side[i] = wi->right_fixed[2*i+1]; 
    }
}
\end{lstlisting}
\end{algorithm}

Com o uso de \textit{loop unswitching} o tempo médio gasto na computação passou
de XXX \textit{ms} para YYY \textit{ms}, este ganho deve ao fato de não ser
realizada a computação da declaração \textit{if} a cada iteração do
\textit{loop}.

%-------------------------------------------------------------------------------

Analisando o algoritmo~\ref{convert_double_1} que apresenta dois \textit{loops},
sendo o primeiro converte um vetor \textit{double} em um vetor 
\textit{short int} e o segundo \textit{loop} combina os \textit{bytes} do vetor
\textit{short} em um vetor \textit{unsigned char}.

Como já visto anteriormente nos algoritmos~\ref{read_data_1},~\ref{fix_data_1} e
~\ref{back_data_1} 
o algoritmo~\ref{convert_double_1} contém uma declaração \textit{if} no corpo de
cada \textit{loop} que também são independentes do respectivo \textit{loop}. 
Assim, será também aplicada a técnica \textit{loop unswitching} em cada um dos
\textit{loops} do algoritmo~\ref{convert_double_1}.


Analisando o algoritmo~\ref{convert_double_2}, nota-se que os 
\textit{loops} tem como
espaço de iteração o valor de \textit{wi->nb\_samples}, embora o segundo
\textit{loop} itere sobre \textit{wi->nb\_samples * iterator},  é possível aplicar
a técnica \textit{loop fusion} nos \textit{loops}, assim unindo-os em um
único \textit{loop}, sendo necessário ajustar o espaço de iteração. 
Devido ao
segundo \textit{loop} utiliza os dados computados pelo primeiro \textit{loop},
será necessário garantir que essa dependência seja mantida.

A aplicação do \textit{loop fusion} sobre o
algoritmo~\ref{convert_double_2} pode se tornar confusa devido a aplicação do
\textit{loop unswitching}, que torna mais complexa a leitura do código. O
algoritmo~\ref{convert_double_3} mostra o 
resultado de \textit{loop fusion} no algoritmo~\ref{convert_double_2}. 


\begin{algorithm}
\caption{Loop com declaração if }
\label{convert_double_1}
\begin{lstlisting}[language=c]
for(i = 0; i < wi->nb_samples; i++){
    wi->short_left[i] = (short int)wi->left_side[i];
    if(wi->wav_header->num_channels == 2){
        wi->short_right[i] = (short int)wi->right_side[i];
    }
}
int iterator = wi->wav_header->num_channels * 2;
int count = 0;
for(i = 0; i < wi->nb_samples * iterator; i += iterator){
    memcpy(&wi->buffer[i], &wi->short_left[count], 2);
    if(wi->wav_header->num_channels == 2){
        memcpy(&wi->buffer[i + 2], &wi->short_right[count], 2);
    }
    count++;
}
\end{lstlisting}
\end{algorithm}

O algoritmo~\ref{convert_double_2} mostra o resultado de \textit{loop
unswitching} no algoritmo~\ref{convert_double_2}.

\begin{algorithm}
\caption{Loops sem declaração if}
\label{convert_double_2}
\begin{lstlisting}[language=c]
if(wi->wav_header->num_channels == 1){
    for(i = 0; i < wi->nb_samples; i++){
        wi->short_left[i] = (short int)wi->left_side[i];
    }
}
else if(wi->wav_header->num_channels == 2){
    for(i = 0; i < wi->nb_samples; i++){
        wi->short_left[i] = (short int)wi->left_side[i];
        wi->short_right[i] = (short int)wi->right_side[i];
    }
}
int iterator = wi->wav_header->num_channels * 2;
int count = 0;
if(wi->wav_header->num_channels == 1){
    for(i = 0; i < wi->nb_samples * iterator; i += iterator){
        memcpy(&wi->buffer[i], &wi->short_left[count], 2);
        count++;
    }
}
if(wi->wav_header->num_channels == 2){
    for(i = 0; i < wi->nb_samples * iterator; i += iterator){
        memcpy(&wi->buffer[i], &wi->short_left[count], 2);
        memcpy(&wi->buffer[i + 2], &wi->short_right[count], 2);
        count++;
    }
}
\end{lstlisting}
\end{algorithm}

\begin{algorithm}
\caption{Loop com declaração if }
\label{convert_double_3}
\begin{lstlisting}[language=c]
if(wi->wav_header->num_channels == 1){
    for(i = 0; i < wi->nb_samples * iterator; i += iterator){
        wi->short_left[i/4] = (short int)wi->left_side[i/iterator];
        memcpy(&wi->buffer[i], &wi->short_left[count], 2);
        count++;
    }
}
if(wi->wav_header->num_channels == 2){
    for(i = 0; i < wi->nb_samples * iterator; i += iterator){
        wi->short_left[i/iterator] = (short int)wi->left_side[i/iterator];
        wi->short_right[i/iterator] = (short int)wi->right_side[i/iterator];
        memcpy(&wi->buffer[i], &wi->short_left[count], 2);
        memcpy(&wi->buffer[i + 2], &wi->short_right[count], 2);
        count++;
    }
}
\end{lstlisting}
\end{algorithm}



Com o uso de \textit{loop unswitching} o tempo médio gasto na computação passou
de XXX \textit{ms} para YYY \textit{ms}, este ganho deve ao fato de não ser
realizada a computação da declaração \textit{if} a cada iteração do
\textit{loop}.




%\begin{algorithm}
%\caption{Loop com declaração if }
%\label{read_data_2}
%\begin{lstlisting}[language=c]
%
%\end{lstlisting}
%\end{algorithm}


