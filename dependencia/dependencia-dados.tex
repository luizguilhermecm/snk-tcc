% vim: set ai tw=80 fileencoding=utf8: 
%-------------------------------------------------------------------------------
\subsection{Dependência de Dados}

Em dependência de dados deve-se garantir que um dado seja produzido e consumido
na ordem correta, assim, cuidando para que não se intercale o \textit{load} e o
\textit{store} em um mesmo local da memória, desta forma, o próximo
\textit{load} pode obter um valor errado. 
Da mesma forma, dois \textit{stores} devem ocorrer na ordem correta para 
que no próximo \textit{load} seja obtido o valor correto.
Assim, dependência de dados pode ser definida como \cite[2.2]{ocfma}:

\begin{description}
        \item [Definição 1:] Existe dependência de dados da declaração $1$ para a declaração 
        $2$ (declaração $2$ depende da declaração $1$) se e somente se:
\begin{enumerate}
        \item Ambas as declarações acessem o mesmo local de memória e ao menos
                umas delas realizará \textit{store} na memória, e
        \item Existe um caminho de execução possível\footnote{O caminho para a
                        execução destas declações deve ser factível durante a 
                execução.} de $1$ para $2$.
\end{enumerate}
\end{description}

Neste capítulo serão apresentadas várias propriedades onde as dependências 
podem ser classificadas.


\subsubsection{Classificação de \textit{Load-Store}}

Em termos da ordem de \textit{load-store}, as dependências podem ocorrer de 
três formas em um programa \cite{Chang:2004}:

\begin{alineas}
\item \textit{True dependence} ou \textit{flow dependence}. Uma declaração realiza 
        \textit{store} em um local da memória em que será realizado um 
        \textit{load} por uma segunda declaração.
        \begin{algorithmic}[1]
        \STATE foo = \ldots
        \STATE \ldots = foo
        \end{algorithmic}
        A dependência garante que $2$ irá ler exatamente o que foi computado
        por $1$. Esse tipo de dependência é também conhecida por dependência 
        de fluxo e é denotada por  $1 \;\delta\; 2$ ou $1 \;\delta^f \; 2$ 
        (lê-se, $2$ depende de $1$).

        \item \textit{Anti-dependence.} Uma primeira declaração realiza
        \textit{load} de um local onde uma segunda declaração irá escrever.
        \begin{algorithmic}[1]
        \STATE \ldots = foo
        \STATE foo = \ldots
        \end{algorithmic}
        Esta dependência previne a troca na ordem de execução entre $1$ 
        e $2$, tal qual poderia resultar em $1$ utilizando-se do valor 
        computado por $2$. 
        Em essência essa dependência existe para prevenir uma transformação que 
        introduziria uma nova dependência do tipo \textit{true dependence} que 
        de fato não existe no programa original. 
        \textit{Anti-dependence} é denotado $1 \;\delta^- \;2$, 
        $1 \;\delta^{-1} \;2$ ou $1 \;\delta^{a} \;2$.

        \item \textit{Output dependence.} Ocorre quando duas declaração 
        realizam \textit{store} em um mesmo local.
        \begin{algorithmic}[1]
        \STATE foo = \ldots
        \STATE foo = \ldots
        \end{algorithmic}
        Essa dependência previne que ocorra uma troca entre as declarações e 
        faça com que uma declaração que irá realizar \textit{load} do valor 
        computado não leia o valor errado.
        \begin{algorithmic}[1]
        \STATE foo = 1
        \STATE \ldots
        \STATE foo = 2
        \STATE bar = 2 * foo
        \end{algorithmic}
        \textit{Output dependence} é denotado  $1 \;\delta^0 \;2$
        ou $1 \;\delta^o \;2$.
\end{alineas}

%-------------------------------------------------------------------------------
\subsubsection{Dependência em Loops}

Estendendo o conceito de dependência para \textit{loops} requer alguma forma de
parametrizar as declarações pelas iterações do \textit{loop} que são
executadas.
Para o \textit{loop} simples do algoritmo~\ref{loop_simples}:

\begin{algorithm}
        \caption{Loop simples}
        \label{loop_simples}
        \begin{algorithmic}[1]
                \FOR {I = 1 to I < N}
                \STATE A(I + 1) = A(I) + B(I) 
        \ENDFOR
        \end{algorithmic}
\end{algorithm}

A declaração $2$ em qualquer iteração do \textit{loop} depende dela mesmo da
iteração anterior. Embora uma simples alteração no \textit{index} do vetor pode
fazer com que a declaração tenha dependência de duas iterações anteriores.
Ver algoritmo~\ref{loop_simples_2}.

\begin{algorithm}
        \caption{Loop simples com dependência de duas iterações}
        \label{loop_simples_2}
        \begin{algorithmic}[1]
                \FOR {I = 1 to I < N}
                \STATE A(I + 2) = A(I) + B(I) 
        \ENDFOR
        \end{algorithmic}
\end{algorithm}

Para melhor definir dependências em \textit{loops},  pode se fazer necessário o
uso de alguma forma de parametrização das declarações, para que através desta
representação seja possível identificar em qual iteração as declarações ocorrem.
Para isso faz-se necessário o uso de números inteiros auxiliares (ou um vetor de
números inteiros), os quais irão representar o número da iteração de cada 
\textit{loop} onde as declaração estão aninhadas \cite{hpcfpc}.
Para o \textit{loop} simples do algoritmo~\ref{loop_1}

\begin{algorithm}
        \caption{Loop simples}
        \label{loop_1}
        \begin{algorithmic}[1]
                \FOR {I = 1 to I < N}
                \STATE \ldots
                \ENDFOR
        \end{algorithmic}
\end{algorithm}

o número da iteração é igual ao número do \textit{loop index}, neste caso o $I$.
O \textit{index} inicia em 1 na primeira iteração, assumindo o valor 2 na
segunda iteração e assim por diante.

Considerando o \textit{loop} parametrizado do algoritmo~\ref{loop_param}

\begin{algorithm}
        \caption{Loop parametrizado}
        \label{loop_param}
        \begin{algorithmic}[1]
                \FOR {I = L to I < U with I = I + S}
                \STATE \ldots
                \ENDFOR
        \end{algorithmic}
\end{algorithm}

\noindent o número da iteração será 1 quando $i$ for igual a $L$, será 2 quando $I$ for
igual a $L+S$ e assim em diante. 
Formalizando-se o conceito de parametrização \cite[2.2]{ocfma};

\begin{description}
        \item [Definição 2:] Para um loop arbitrário em que o index $I$ do loop é
        incrementado de $L$ até $U$ em incrementos de $S$, o número da iteração
        (normalizado) $i$ de uma iteração qualquer terá o valor de 
        $(I - L + 1)/ S$, onde $I$ é o valor do index daquela iteração.
\end{description}

No caso de \textit{loops} aninhados, o nível de aninhamento de um determinado 
\textit{loop} será igual ao número de \textit{loops} que estão iterando sobre
ele somado de um. Em aninhamento de \textit{loops} os mesmo serão enumerados do
mais externo para o mais interno, começando em 1. Formalizando-se este conceito
tem-se \cite[2.2]{ocfma};

\begin{description}
        \item[Definição 3:] Para um aninhamento de $n$ loops, o vetor de
                iteração $i$ de uma determinada iteração no loop mais interno 
                é um vetor de inteiros que contém o número da iteração de cada 
                loop na ordem do aninhamento. 
                O vetor de iterações é dado pela seguinte equação
                $i = {i_1, i_2, \ldots, i_n}$
                onde $i_k, 1 \leq k \leq n$, representa o número da iteração do 
                k-ésimo loop aninhado.
\end{description}

Considerando-se um aninhamento de dois \textit{loops} e o vetor de iteração
$it\_array[2]$, onde $it\_array[0]$ contém o número da iteração do \textit{loop}
mais externo e $it\_array[1]$ do loop mais interno. 
O conjunto de todas as possibilidades do vetor de iteração é chamado de espaço 
de iteração (\textit{iteration space}). 
O espaço de iteração para este aninhamento seria $\{(1,1),(1,2),(2,1),(2,2)\}$.

Devido a importância da ordem de execução para tratamento de dependência,
vetores de iteração precisam ser precisos quanto a ordem dos seus elementos
internos em relação a ordem de aninhamento dos \textit{loops}. 










%referencia: ocfma page 60
%-------------------------------------------------------------------------------
