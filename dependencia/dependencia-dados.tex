% vim: set ai tw=80 fileencoding=utf8: 
%-------------------------------------------------------------------------------
\section{Dependência de Dados}

Em dependência de dados deve-se garantir que um dado seja produzido e consumido
na ordem correta, assim, cuidando para que não se intercale a leitura e
escrita em um mesmo local, desta forma, durante a leitura pode-se obter um valor 
errado. Da mesma forma, duas escritas devem ocorrer na ordem correta para que na
próxima leitura seja obtido o valor correto.
Dependência de dados pode ser definida como:

\begin{verbatim}
Definição 1: Existe dependência de dados da declaração S1 para a declaração S2
(declaração S2 depende da declaração S1) se e somente se:
        1 -> Ambas as declarações acessam o mesmo local de memória e ao menos
        umas delas escreverão na memória, e
        2 -> Existe um caminho de execução viável de S1 para S2.
\end{verbatim}

Neste capítulo serão apresentadas várias propriedades em que as dependências 
podem ser classificadas.


\subsection{Classificação de Leitura-Escrita}

Em termos da ordem de leitura-escrita, as dependências podem ocorrer de três
maneiras em um programa:

\begin{alineas}
        \item \textit{True dependence.} Onde uma declaração escreve em um local
        que será lido por uma segunda declaração.
        \begin{verbatim}
        S1    x = ...
        S2    ... = x
        \end{verbatim}
        A dependência garante que $S_2$ irá ler exatamente o que foi computado
        por $S_1$. Esse tipo de dependência é também conhecida por dependência de
        fluxo e é denotada por $S_1 \delta S_2$ (lê-se, $S_2$ depende de $S_1$)

        \item \textit{Antidependence.} Uma primeira declaração le de um local
        onde uma segunda declaração irá escrever.
        \begin{verbatim}
        S1    ... = x
        S2    x = ...
        \end{verbatim}
        Esta dependência previne que a troca entre $S_1$ e $S_2$, qual poderia
        fazer com que $S_1$ utiliza-se o valor computado por $S_2$. Em essência
        essa dependência existe para prevenir uma transformação que introduziria
        uma nova dependência do tipo \textit{true dependence} que de fato não
        existe no programa original. \textit{Antidependence} é denotado $S_1
        \delta^- S_2$ ou $S_1 \delta^{-1} S_2$.

        \item \textit{Output dependence.} Ocorre quando duas declaração escrevem
        em um mesmo local.
        \begin{verbatim}
        S1    x = ...
        S2    x = ...
        \end{verbatim}
        Essa dependência previne que ocorra uma troca em as declarações e faça com
        que uma declaração que irá ler o valor computado não leia o valor
        errado.
        \begin{verbatim}
        S1    x = 1
        S2    ...
        S3    x = 2 
        S4    y = 2 * x
        \end{verbatim}
        \textit{Output dependence} é denotado $S_1 \delta^0 S_2$.
\end{alineas}






%referencia: ocfma
%-------------------------------------------------------------------------------
