% vim: set ai tw=80 fileencoding=utf8: 
%-------------------------------------------------------------------------------

\section{Depêndencia}

Quando um programador escreve um programa em uma linguagem sequêncial, o
resultado esperado será obtido pela execução da primeira linha, depois a segunda
e assim em diante, considerando exceções de controles de fluxos como
\textit{loops} e ramificações. 
Uma vez que o programador especificou a ordem que ele espera que as computações 
sejam realizadas. 
Obter paralelismo de um programa respeitando a estas especificações não é
possível, uma vez que obter paralelismo implica em alterar a ordem das
operações realizadas.

Paralelizar um programa sequêncial significa encontrar uma ordem de execução
diferente da especificada e que irá sempre computar o mesmo resultado.
A programação sequêncial introduz restrições que podem ser críticas para o
resultado esperado do programa, assim para transformar um programa em paralelo é
importante encontrar as restrições menos críticas e realizar transformações para
que o programa continue retornando o resultado correto para qualquer entrada
\cite[1.8]{ocfma}.

%Neste capítulo serão apresentadas uma série de restrições, chamadas de
%dependências que serão necessárias para garantir que as transformações
%realizadas nos programas não afetem o resultado e o significado das computações
%realizadas pelo programa.

Uma dependência é uma relação entre duas declarações no programa. 
Um par de declarações $<1,2>$ está em uma relação se $2$ é executada 
depois de $1$ em um programa sequêncial, e deve ser executada após $1$ 
em qualquer reordenação válida do programa se a ordem de acesso a 
memória será preservada.

\begin{algorithm}
        \begin{algorithmic}[1]
                \STATE PI = 3.14159
                \STATE R = 5
                \STATE AREA = PI * R * R
\end{algorithmic}
\end{algorithm}


Os resultados obtidos por estas declarações são definidas por aqueles obtidos
quando a ordem da execução realizada seja $<1,2,3>$. 
No entanto nada neste trecho de código torna obrigatório a execução de 
$2$ depois de $1$, desta forma, os resultados obtidos pela ordem de execução 
$<2,1,3>$ serão os mesmos para a variável $AREA$, seja qual for o valor 
da entrada.
Por outro lado, o momento de execução de $3$ é mais crítico, se $3$ for
executada antes de $1$ ou de $2$, os resultados obtidos por esta execução
diferenciariam dos resultados obtidos das computações realizadas na ordem
original.
Em termos de dependência pode-se observar que os pares $<1,3>$ e $<2,3>$
estão em uma relação de dependência, embora o par $<1,2>$ não.
Dependências desse tipo são ditas dependência de dados.

Dependência em linhas de código sequêncial como visto anteriormente, é um
conceito simples de entender.
O problema é que examinar somente linhas de códigos sequênciais não garante 
eficiência em termos de paralelismo. 
Para se obter tal eficiência deve-se considerar os trechos de
código que são mais executados, ou seja, devemos expandir o conceito de
dependência para \textit{loops} e vetores.
O trecho de código a seguir ilustra a complexidade introduzida ao expandirmos o
conceito de dependência.

\begin{algorithm}
        \begin{algorithmic}[1]
                \FOR {I = 1 to I < N}
                \STATE A(I) = B(I) + 1
                \STATE B(I + 1) = A(I) - 5
                \ENDFOR
        \end{algorithmic}
\end{algorithm}

Este \textit{loop} mostra a dependência entre $<2,3>$, uma vez que o
resultado computado de $A$ é imediatamente utilizando por $3$ em todas as
iterações, e também a dependêndia entre $<3,2>$ exceto na primeira iteração,
uma vez que o resultado obtido por $3$ será utilizado na iteração anterior.
Detectar estas dependências é difícil, considerando que cada iteração acessa 
diferentes elementos do vetor.

\textit{Loops} e vetores são apenas parte do problema que envolve a análise de
dependência, deve-se considerar também as estruturas condicionais, como as
declarações $IF$.

Deve-se então entender um outro tipo de dependência, dado o trecho de
código a seguir:

\begin{algorithm}
        \begin{algorithmic}[1]
        \IF{D \!= 0}
        \STATE A = A / D
        \ENDIF
        \end{algorithmic}
\end{algorithm}

A declaração $2$ não pode ser executada antes de $1$, uma vez que essa
transformação pode ocasionar em uma divisão por zero, o que não ocorreria no
programa original. Essa dependência é chamada de dependência de controle.


%referencia: ocfma
%-------------------------------------------------------------------------------


