% vim: set ai tw=80 fileencoding=utf8: 
%-------------------------------------------------------------------------------

\chapter{Depêndencia}

Quando um programador escreve um programa em uma linguagem sequêncial, o
resultado esperado será obtido pela execução da primeira linha, depois a segundo
e assim em diante, considerando exceções de controles de fluxos como
\textit{loops} e ramificações. 
Uma vez que o programador especificou a ordem que ele espera que as computações 
sejam realizadas. 
Obter parelelismo de um programa respeitando a estas especificações não é
possível, uma vez que obter paralelismo implica em alterar a ordem das
operações realizadas.

Paralelizar um programa sequêncial significa encontrar uma ordem de execução
diferente da especificada e que irá sempre computar o mesmo resultado.
A programação sequêncial introduz restrições que podem ser críticas para o
resultado esperado do programa, assim para transformar um programa em paralelo é
importante encontrar as restrições menos críticas e realizar transformações para
que o programa continue retornando o resultado correto para qualquer entrada.

Neste capítudo serão apresentadas uma série de restrições, chamadas de
dependências que serão necessárias para garantir que as transformações
realizadas nos programas não afetem o resultado e o significado das computações
realizadas pelo programa.

Uma dependência é uma relação entre duas declarações no programa. Um par de
declarações $<S_1,S_2>$ está em uma relação se $S_2$ é executada depois de $S_1$
em um programa sequêncial, e deve ser executada após $S_1$ em qualquer
reodenação válida do programa se a ordem de acesso a memória será preservada.

\begin{verbatim}
S1   PI = 3.14159
S2   R = 5 
S3   AREA = PI * R * R
\end{verbatim}

%referencia: ocfma
%-------------------------------------------------------------------------------


