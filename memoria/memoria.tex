% vim: set ai tw=80 fileencoding=utf8: 
%-------------------------------------------------------------------------------
\chapter{Memória}

A mais simples maneira de se melhorar o desempenho de um sistema é
replicar os computadores e criar uma forma destes trocarem dados.
Desta forma consegue-se aumentar o desempenho sem que seja necessário alterar a
\textit{CPU}.

Com o aumento contínuo da necessidade de desempenho em aplicações cada vez mais
custosas a maioria dos sistemas paralelos utilizam-se de uma entre duas
tecnologias, memória distribuída ou memória compartilhada.


%-------------------------------------------------------------------------------
\section{Memória Distribuída}

Memória distribuída ou \textit{distributed memory} ou \textit{shared-nothing} é
a mais simples abordagem do ponto de vista do \textit{hardware}. 
A premissa desta abordagem é utilizar vários computadores interligados através 
de uma rede.

O modelo padrão de programação consiste de processos separados para cada
computador que se comunicam através da troca de mensagem ou 
\textit{message passing}, o que normalmente é feito através de bibliotecas
desenvolvidas com esse propósito. 
Sendo este é o modelo mais clássico de computação paralela. 
A forma moderna de sistemas com memória distribuída iniciou a partir do trabalho 
de Seitz em 1985 \cite{Seitz:1985}.

Devido ao baixo custo de processadores voltados ao mercado consumidor e da fácil
montagem, alguns grupos exploraram tais fatores começaram a construir
\textit{cluster} de computadores pessoais. 
Tais \textit{clusters} já chamados de \textit{NOWs}, 
\textit{Network of Workstations}.
Combinando todos estes fatores com o rápido avanço de desempenho de computadores
pessoais e o avanço do \textit{open-source} junto com  versões de sistemas 
operacionais UNIX ajudaram a difundir sistemas com tais caracteristicas. 
Hoje estes sistemas são comumente conhecidos como \textit{Beowulfs} ou
\textit{Beowulf Cluster} devido ao projeto de Thomas Sterling e Donald Becker
realizado na NASA.

%-------------------------------------------------------------------------------
\section{Memória Compartilhada}

Memória compartilhada ou \textit{shared memory} é uma abordagem mais complexa, 
tornando a memória visível a todos os processadores, permitindo que 
todos possam carregar e gravar do mesmo endereço de memória. 

Entre as dificuldades desta abordagem os dois que chamam mais atenção são
coerência e consistência.
Sendo a consistência o mais problemático para o programador.




% ISA instruction set architecture

%referencias:
%sopc
