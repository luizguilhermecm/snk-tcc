% vim: set ai tw=80 fileencoding=utf8: 
%-------------------------------------------------------------------------------
\section{Taxonomia de Flynn}

Taxonomia de Flynn, proposta em 1966 \cite{Flynn:1966} por Michael 
Flynn e expandida em 1972 \cite{Flynn:1972}, é uma das formas de classificar o 
paralelismo disponível no processador.  

Taxonomia de Flynn utiliza o conceito de sequência de objetos ou ações, que são
chamados de \textit{stream}.
Flynn introduziu dois tipos de \textit{stream}, o 
\textit{stream} de instrução e também o \textit{stream} de dados. 

O \textit{stream} de instrução consiste em uma sequência de instruções. 
Uma instrução ou \textit{instruction word (IW)} é uma cadeia de 0's e 1's que 
representa a menor operação visível ao programador e que será executada pelo 
processador. 
Uma instrução pode conter uma ou mais operações, devido a esta peculiaridade,
alguns autores utilizam \textit{instruction} para instruções que contenham 
apenas uma operação e \textit{instruction word} para instruções que contenham 
mais de uma operação.

\begin{comment}
Processadores escalares (\textit{scalar processors}) e processadores
superescalares (\textit{superscalar processors}) executam uma ou mais
\textit{instructions} por ciclo de \textit{clock} da máquina. 
\end{comment}

Existem, no entanto, quatro combinações de \textit{streams} que descrevem as 
arquiteturas de computadores mais comuns \cite{Flynn:1996}:

\begin{enumerate}
        \item \textbf{SISD:} \textit{Single Instruction, Single Data}
        \item \textbf{SIMD:} \textit{Single Instruction, Multiple Data}
        \item \textbf{MISD:} \textit{Multiple Instruction, Single Data}
        \item \textbf{MIMD:} \textit{Multiple Instruction, Multiple Data}
\end{enumerate}

Cada combinação de \textit{streams} caracteriza uma classe de arquitetura 
e cada uma destas classes possui seus tipos de paralelismo específicos.


%-------------------------------------------------------------------------------
\subsection{SISD: Single Instruction, Single Data}

A classe de arquiteturas de processadores \textit{SISD}, inclui a 
maior parte dos processadores ainda em uso na data da publicação deste trabalho, 
os processadores 
\textit{single-core}, embora os programadores não percebam o paralelismo 
inerente destes, muita concorrência pode estar presente.  

Em 1966 Flynn cita em \cite{Flynn:1996} o \textit{pipeline} como uma forma de 
se obter concorrência 
nos processadores \textit{SISD}, embora Flynn considere a decodificação das 
inúmeras \textit{instructions} como sendo um \textit{bottleneck}, devido à 
tecnologia da época. 
Na data de publicação deste trabalho, grande parte dos processadores 
utilizam-se de \textit{pipeline} assim como também se aproveitam de alguma forma 
de múltiplas \textit{instructions}.

A concorrência em processadores \textit{SISD} é explorada em tempo de execução,
realizando mais de uma operação por ciclo de \textit{clock} da máquina.
%do \textit{stream} de instruções.

A quantidade e o tipo de paralelismo possível em processadores \textit{SISD}
é determinada por quatro fatores principais:

\begin{enumerate}
        \item O número de operações que podem ser executadas concorrentemente.
        \item A forma como as operações serão arranjadas para execução,
                podendo ser estaticamente, dinamicamente ou até mesmo utilizando
                ambas.
        \item A ordem em que as operações são colocadas e retiradas em relação
                a ordem original do programa.
        \item A maneira como o processador irá tratar cada exceção, podendo ser: 
                preciso impreciso ou ambos.
\end{enumerate}

%-------------------------------------------------------------------------------
\subsubsection{Processadores Escalares}

Processadores escalares são processadores simples, que executam no 
máximo uma instrução e no máximo uma operação por ciclo de \textit{clock} de 
máquina \cite[3.5]{aapc}. 
As instruções do \textit{stream} de instruções são executadas 
sequencialmente, assim uma nova instrução não será executada até que a execução 
da instrução em execução seja finalizada e seu resultado devidamente
armazenado.
A semântica de instrução determina que uma sequência de ações devem ocorrer
para que se obtenha o resultado esperado, sendo: buscar a instrução,
decodifica-la, acessar o dado ou registrador, execução da operação e armazenar o
resultado. 
Podendo ocorrer \textit{overlap} entre as ações, mas o resultado deve
aparecer na ordem especificada.
Esse comportamento sequencial descreve o modelo de execução sequencial.
No modelo de execução sequencial, a execução é dita \textit{instruction-precise}
se encontrar as seguintes condições \cite{eopc}:

\begin{itemize}
        \item Todas as instruções ou operações que precederam a instrução atual
                ou operação atual já foram executadas e seus resultados
                armazenados.
        \item Todas as instruções ou operações na fila de execução não foram
                executadas ou seus resultados ainda não foram armazenados.
        \item A instrução ou operação em execução no momento está em um dos
                estados de execução, tendo ou não seu resultado já armazenado.
\end{itemize}

A maioria dos processadores escalares implementam diretamente o modelo de
execução sequencial.


%-------------------------------------------------------------------------------
\subsubsection{Processadores Superescalares}

Enquanto processadores escalares estão limitados a executar uma única instrução 
por ciclo de \textit{clock} de máquina os processadores superescalares decoficam
várias instruções por ciclo de \textit{clock} de máquina, utilizando várias
unidades funcionais e alocação dinâmica para executar várias instruções por
ciclo de \textit{clock} de máquina. 
Processadores superescalares têm um comportamento similar ao \textit{pipeline}
\cite{eopc}.

A capacidade de executar múltiplas instruções implica em verificar se existe
dependências entre as instruções, essa verificação tem que ser feita em nível de
\textit{hardware}. 
Processadores superescalares mais avançados geralmente incluem 
\textit{hardwares} que preservam a ordem e precisamente lida com as exceções, 
assim simplificando o modelo de programação.

Devido a complexidade da lógica para alocação dinâmica das instruções,
processadores superescalares de alto desempenho, em geral estão limitados a
executarem de quatro a oito instruções por ciclo de \textit{clock} de máquina.


%-------------------------------------------------------------------------------
\subsubsection{Processadores VLIW}

Processadores VLIW (\textit{Very Long Instruction Word}) assim como os
processadores superescalares decodificam inúmeras instruções por ciclo de
\textit{clock} de máquina e utilizam várias unidades funcionais.

Ao contrário dos processadores superescalares que utilizam \textit{hardware}
para realizar alocação dinâmica das instruções, os processadores VLIW executam as 
instruções através de alocação estática, esta alocação depende de uma análise do
compilador.
Assim, os processadores VLIW são menos complexos e apresentam desempenho 
potencialmente maior \cite{eopc}.

Processadores VLIW apresentam grande desempenho em aplicações que podem ser 
efetivamente alocadas de forma estática, embora nem todas as aplicações
tenham esta característica, assim, a ordem de execução estática determinada pelo
compilador não procede. 
Duas classes de execução podem ocorrer e afetar o comportamento da execução 
estática:

\begin{enumerate}
        \item Atrasos de resultados das operações, devido à diferença da latência
                ocorrida com a latência considerada durante a alocação pelo
                compilador.
        \item Exceções ou interrupções que colocam a ordem de execução em um
                estado não antecipado pelo compilador.
\end{enumerate}

O processador consegue lidar com atrasos, embora isso tenha um impacto 
significante no desempenho. 
A causa mais comum de atrasos na execução devem ao dado não estar mais na 
memória cache, esse fator é tratado considerando-se o pior caso de latência 
possível e até evitando o uso da memória cache. 
Na falta de paralelismo para cobrir as lacunas da latênica, resulta na 
alocação de instruções com um número menor de operações que o processador 
consegue executar, assim diminuindo o desempenho \cite{ocfma}.


%-------------------------------------------------------------------------------
\subsection{SIMD: Single Instruction, Multiple Data}

A classe de processadores \textit{SIMD} incluem dois tipos de
processadores, vetoriais e matriciais.
Processadores \textit{SIMD} são projetados para utilizarem determinadas
estruturas de dados, como vetores e matrizes. 
Em nível de código de máquina, programar para processadores \textit{SIMD} é 
bastante similar a processadores \textit{SISD}, a diferença é realizar operações
nas estruturas de dados agregadas. Como no processamento de algoritmos 
científicos há um grande uso de vetores e matrizes, processadores \textit{SIMD}
têm obtido grande desempenho na área \cite{eopc}.

Processadores vetoriais e matriciais apresentam diferenças tanto na 
implementação quanto na organização dos dados.

Processadores matriciais consistem em elementos de processos interconectados,
cada um tendo seu próprio espaço de memória. Processadores vetoriais consistem
em um único processo que referencia a um espaço de memória global.

%-------------------------------------------------------------------------------
%\subsubsection{Processadores Matriciais}


%-------------------------------------------------------------------------------
%\subsubsection{Processadores Vetoriais}


%-------------------------------------------------------------------------------
\subsection{MISD: Multiple Instruction, Single Data}

A classe de processadores \textit{MISD} abstratamente é um
\textit{pipeline} de múltiplas unidades funcionais operando independentemente
sob um único \textit{stream} de dados. Em nível de micro-arquitetura é
exatamente o que os processadores vetoriais fazem \cite[2.3]{aipp}.

Segundo \cite{Openshaw:1999} exceto no caso de um cientista da computação
interessado em estranhas formas de computação a classe \textit{MISD} é uma forma
restritiva e impraticável de paralelismo.

%-------------------------------------------------------------------------------
\subsection{MIMD: Multiple Instruction, Multiple Data}

Na classe de processadores \textit{MIMD} estão os multiprocessadores
com alguma forma de interconexão entre os processadores. Do ponto de vista do
programador, cada processo é executado independentemente e de forma cooperativa
para solucionar um mesmo problema, embora alguma forma de sincronização entre os
processos é necessária para que as informações e dados sejam trocados entre os
processadores \cite{eopc}.

Não existem limitações em que todos os processadores sejam idênticos, embora a
maioria das configurações \textit{MIMD} sejam homogêneas com processadores
idênticos. Configurações heterogêneas de processadores são geralmente utilizados
para aplicações com propósitos específicos.

Da perspectiva do \textit{hardware} existem dois tipos de \textit{MIMD}, sendo os
processadores \textit{multi-cores} e processadores \textit{multi-threaded}.


%-------------------------------------------------------------------------------
\subsubsection{Processadores Multi-Threaded}

Em \textit{multi-threaded} \textit{MIMD}, um processador base é
estendido para incluir múltiplos conjuntos de registradores para dados e para o
programa.
Com essa configuração, diferentes \textit{threads} ou programas ocupam cada
conjunto de registrador. Assim que recursos se tornam disponíveis as
\textit{threads} continuam sua execução \cite{Flynn:1996}.

Uma vez que cada \textit{thread} é independente, também o é no uso dos recursos
disponíveis, assim múltiplas \textit{threads}, fazem melhor uso dos recursos e
em consequência aumenta-se o número de instruções executadas por ciclo de
\textit{clock} de máquina.

\textit{Threads} ditas críticas podem ter prioridade na execução para garantir
que sejam executadas em menor tempo, enquanto \textit{threads} não críticas se
utilizam de recursos ociosos.


%-------------------------------------------------------------------------------
\subsubsection{Processadores Multi-Core}

Os processadores \textit{multi-core} e também os múltiplos
\textit{multi-core} necessitam comunicar os resultados de suas execuções
através de uma rede de intercomunicação e de controle de tarefas.
Assim sua implementação é significativamente mais complexa que processadores
\textit{multi-threaded}.

A rede de intercomunicação de troca de dados entre os processadores realiza a
sincronização das execuções independentes.

Quanto a comunicação realizada entre os processadores através de memória
compartilhada surgem dois principais problemas, manter a consistência da memória
e também a coerência de cache.
A solução para ambos os problemas se dão em técnicas de \textit{software} e
\textit{hardware} \cite{eopc}.
