% vim: set ai tw=80 fileencoding=latin1: 
%-------------------------------------------------------------------------------
\chapter{Mem�ria}

\noindent A mais simples maneira de se melhorar o desempenho de um sistema �
replicar os computadores e criar uma forma destes trocarem dados.
Desta forma consegue-se aumentar o desempenho sem que seja necess�rio alterar a
\textit{CPU}.

Com o aumento cont�nuo da necessidade de desempenho em aplica��es cada vez mais
custosas a maioria dos sistemas paralelos utilizam-se de uma entre duas
t�cnologias, mem�ria distribu�da ou mem�ria compartilhada.


%-------------------------------------------------------------------------------
\subsection{Mem�ria Distribu�da}
\noindent
Mem�ria distribu�da ou \textit{distributed memory} ou \textit{shared-nothing} �
a mais simples abordagem do ponto de vista do \textit{hardware}. A premissa
desta abordagem � utilizar v�rios computadores interligados atrav�z de uma rede.

O modelo padr�o de programa��o consiste de processos separados para cada
computador que se comunicam atrav�z da troca de mensagem ou 
\textit{message passing}, o que normalmente � feito atrav�z de bibliotecas
desenvolvidas com esse prop�sito. Sendo este � o modelo mais cl�ssico de 
computa��o paralela. A forma moderna de sistemas com mem�ria distribu�da iniciou
apartir do trabalho de Seitz em 1985 \cite{Seitz:1985}.

Devido ao baixo custo de processadores voltados ao mercado consumidor e da f�cil
montagem, alguns grupos exploraram tais fatores come�aram a construir
\textit{cluster} de computadores pessoais. Tais \textit{clusters} j� chamados de
\textit{NOWs}, \textit{Network of Workstations}.
Combinando todos estes fatores com o r�pido avan�o de desempenho de computadores
pessoais e o avan�o do \textit{open-source} junto com  vers�es de sistemas 
operacionais UNIX ajudaram a difundir sistemas com tais caracteristicas. Hoje
estes sistemas s�o comumente conhecidos como \textit{Beowulfs} ou
\textit{Beowulf Cluster} devido ao projeto de Thomas Sterling e Donald Becker
realizado na NASA.

%-------------------------------------------------------------------------------
\subsection{Mem�ria Compartilhada}
