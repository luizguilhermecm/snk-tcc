% vim: set ai tw=80 fileencoding=utf8: 
%-------------------------------------------------------------------------------

\chapter{Data Flow Graph}

\textit{Data Flow Graph} (DFG) ou grafo de fluxo de dados, é um modelo para 
programas que expressa a possibilidade de execução concorrente de partes do
programa. 
Nos DFGs os nós representam operações (funções) e predicados a serem
aplicados a objetos de dados e as arestas representam a ligação entre o nó que
produz o dado e o nó que irá consumir aquele dado.
Na literatura os nós também são chamados de atores.
Assim, aspectos de controle e de dados de um programa podem ser representados 
em um único modelo integrado.

Embora muitas versões de DFGs tenham sido estudadas na literatura, elas possuem 
algumas características em comum:

\begin{alineas}
        \item DFG é um grafo orientado onde uma aresta é um caminho que um dado
        percorre do nó produtor para o nó consumidor.
        \item Dinamicamente, o nó de um DFG aceita um ou mais dado como entrada,
        realizando computações e devolvendo os dados do retorno para suas
        saídas.
        \item Uma ação de um nó é ativada com a presença dos dados de entrada.
\end{alineas}

Os estudos em DFGs tem sido focado principalmente em três modelos bem definidos:
DFGs estáticos, DFGs dinâmicos e DFGs síncronos.

%eopc
%-------------------------------------------------------------------------------
%\section{Subcapítulo}


