% vim: set ai tw=80 fileencoding=utf8: 
%-------------------------------------------------------------------------------

\section{Data Flow Graph}

\textit{Data Flow Graph} (DFG) ou grafo de fluxo de dados, é um modelo para 
programas que expressa a possibilidade de execução concorrente de partes do
programa. 
Nos DFGs os nós representam operações (funções) e predicados a serem
aplicados a objetos de dados e as arestas representam a ligação entre o nó que
produz o dado e o nó que irá consumir aquele dado.
Na literatura os nós também são chamados de atores.
Assim, aspectos de controle e de dados de um programa podem ser representados 
em um único modelo integrado \cite{eopc}.

Embora muitas versões de DFGs tenham sido estudadas na literatura, elas possuem 
algumas características em comum:

\begin{alineas}
        \item DFG é um grafo orientado em que uma aresta é um caminho que um dado
        percorre do nó produtor para o nó consumidor;
        \item Dinamicamente, o nó de um DFG aceita um ou mais dado como entrada,
        realizando computações e devolvendo os dados do retorno para suas
        saídas;
        \item Uma ação de um nó é ativada com a presença dos dados de entrada.
\end{alineas}

Os estudos em DFGs tem sido focado principalmente em três modelos bem definidos:
DFGs estáticos, DFGs dinâmicos e DFGs síncronos.

Um exemplo de DFG pode ser visto na figura~\ref{dfg_ex}, onde o nó 1 é o
produtor de dados para o nó 2 e 4, o nó 2 é produtor de dados para o nó 5, 
o nó 3 é produtor de dados para os nós 4 e 5 e os nós 4 e 5 não são produtores,
apenas consumidores. Entre as ordens de execuções que respeitem o DFG está 
1, 2, 3, 4 e 5, outra execução possível é 3, 1, 4, 2 e 5.

\begin{figure}[h]
\centering
\label{dfg_ex}
\begin{tikzpicture}[->,>=stealth',shorten >=1pt,auto,node distance=1.5cm,
  thick,main node/.style={circle,draw}]

  \node[main node] (1) {1};
  \node[main node] (2) [below of=1] {2};
  \node[main node] (3) [below of=2] {3};
  \node[main node] (4) [below of=3] {4};
  \node[main node] (5) [below of=4] {5};

  \path[every node/.style={font=\sffamily\small}]
    (1) edge [bend right] node [above] {} (4)
        edge [bend right] node [above] {} (2)
    (2) edge [bend left] node [above] {} (5)
    (3) edge node [above] {} (4)
        edge [bend right] node [above] {} (5);

\end{tikzpicture}
\caption{Exemplo de DFG}
\end{figure}


%eopc
%-------------------------------------------------------------------------------
%\section{Subcapítulo}


