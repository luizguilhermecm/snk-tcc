% vim: set ai tw=80: 
%-------------------------------------------------------------------------------
\chapter{Taxonomia de Flynn}

\noindent Taxonomia de Flynn, proposta em 1966 \cite{flynn1966} por Michael 
Flynn e melhorada em 1972 \cite{flynn1972}, é uma das formas de classificar o 
parale lismo disponível no processador.  

%A nível de instrução, paralelismo significa que várias operações podem ser 
%executadas concorrentemente em um programa.  

Taxonomia de Flynn utiliza o conceito de sequência de objetos ou ações, que são
chamados de \textit{Stream}. Há dois tipos de \textit{Stream}, de instrução e 
de dados. Existe no entanto 4 combinação de \textit{Streams} que descrevem as 
arqu iteturas de computadores mais comuns \cite{Flynn1996}:

\begin{itemize}
        \item \textbf{SISD:} \textit{Single Instruction, Single Data}
        \item \textbf{SIMD:} \textit{Single Instruction, Multiple Data}
        \item \textbf{MISD:} \textit{Multiple Instruction, Single Data}
        \item \textbf{MIMD:} \textit{Multiple Instruction, Multiple Data}
\end{itemize}

\subsection{SISD: Single Instruction, Single Data}

\noindent A classe de arquiteturas de processadores \textit{SISD} incluem a 
maior parte dos processadores utilizados nos dias de hoje, os processadores 
\textit{single-core}, embora os programadores não percebam o paralelismo 
inerente destes processadores, muita concorrência pode estar presente.  

Em 1966 Flynn cita o \textit{Pipeline} como uma forma de se obter concorrência 
em processadores \textit{SISD}, embora ele considere a decodificação das 
inúmeras instruções como sendo um \textit{bottleneck} devido a tecnologia da 
época. Nos dias de hoje, grande parte dos dos processadores utilizam-se de 
\textit{Pipeline} assim como também se aproveitam de alguma forma de múltiplas 
instruções.  

A concorrência em processadores \textit{SISD} são explorados durante a execução,
assim realizando mais de uma operação por ciclo de \textit{clock} da máquina 
através do \textit{Stream} de instruções. \textit{Stream} de instruções 
consiste em uma sequência de instruções ou também chamada de 
\textit{instruction word (IW)}

Uma instrução ou \textit{instruction word} é uma cadeia de 0's e 1's que 
representa a menor operação visível ao programador e que será executada pelo 
processador.



\subsection{SIMD: Single Instruction, Multiple Data}



\subsection{MISD: Multiple Instruction, Single Data}



\subsection{MIMD: Multiple Instruction, Multiple Data}
 
