% vim: set ai tw=80 fileencoding=utf8: 
%-------------------------------------------------------------------------------
\section{Loop Interchanging}

\textit{Loop interchanging} talvez seja a mais importante técnica de
reestruturação de \textit{loops}.
Esta técnica resume-se em \textit{loops} que estejam fortemente aninhados
serem trocados de posição, o \textit{loop} mais externo é trocado pelo mais
interno.
No caso de apenas um \textit{loop} carregar todas as dependências, este pode ser
trocado de posição e se tornar o \textit{loop} mais externo, assim, os outros
\textit{loops} podem ser executados em paralelo.

\textit{Loop interchanging} tem sido utilizado para melhorar o desempenho 
de um programa.
Um \textit{loop} mais externo com um número maior de iterações e o 
\textit{loop} mais interno com apenas algumas iterações, em um cenário como este
haverá \textit{loop startup overhead} sobre o \textit{loop} mais interno,
trocando-os de posição, a ocorrência de \textit{loop startup overhead} será de
apenas algumas vezes.

%-------------------------------------------------------------------------------
\subsection{Loops Fortemente Aninhados}

Dois \textit{loops} estão fortemente aninhados quando o único código no corpo do
\textit{loop} mais externo for o \textit{loop} mais interno.
Para \textit{loops} contáveis, com percurso do \textit{loop} mais interno não
dependendo da iteração do \textit{loop} mais externo, aplicar \textit{loop
interchanging} será apenas troca-los de posição.
Ao aplicar \textit{loop interchanging} no algoritmo~\ref{fortemente_ex1}
seria obtido o algoritmo~\ref{fortemente_ex2} como resultado.

\begin{algorithm}
\caption{Dois \textit{loops} fortemente aninhados}
\label{fortemente_ex1}
\begin{algorithmic}[1]

\FOR {I = }
\FOR {J = }
\STATE corpo do loop
\ENDFOR
\ENDFOR

\end{algorithmic}
\end{algorithm}


\begin{algorithm}
\caption{Resultado de \textit{loop interchanging} no
        algoritmo~\ref{fortemente_ex1}}
\label{fortemente_ex2}
\begin{algorithmic}[1]

\FOR {J = }
\FOR {I = }
\STATE corpo do loop
\ENDFOR
\ENDFOR

\end{algorithmic}
\end{algorithm}

Assim como em todas as técnicas de reestruturação de \textit{loops}, deve-se
atentar em preservar as relações de dependência entre os \textit{loops}. 
Se toda dependência for preservada e o sentido do programa se mantiver, então, o
resultado do \textit{loop interchanging} será sempre válido.


%-------------------------------------------------------------------------------
\subsection{Loops Não Fortemente Aninhados}

\textit{Loops} não fortemente aninhados, ao contrário do fortemente aninhado, o
\textit{loop} mais externo tem mais declaração em seu corpo do que apenas o
\textit{loop} interno.
As vezes é  possível  transformar  \textit{loops} não fortemente aninhados em
fortemente aninhados utilizando-se \textit{loop fission}.
Algumas vezes isto pode não ser possível, então outra forma de faze-lo seria
colocar estas declarações do \textit{loop} mais externo dentro do corpo do
\textit{loop} mais interno, utilizando-se de declarações condicionais para
ajustar em qual iteração devem ser executadas. 
Um ajuste possível seria executa-las quando o \textit{index} dos \textit{loops}
forem iguais, outro ajuste seria a criação de uma nova iteração no início ou fim
para executa-las.

Utilizando o ajuste condicional de igualdade  do \textit{index} com mesmo limite
de percurso aplicada ao algoritmo~\ref{nf_1}, resultaria no
algoritmo~\ref{nf_2}.

\begin{algorithm}
\caption{Dois \textit{loops} não fortemente aninhados}
\label{nf_1}
\begin{algorithmic}[1]

\FOR {I = 1 to N}
\STATE corpo do loop externo
\FOR {J = 1 to N}
\STATE corpo do loop interno
\ENDFOR
\ENDFOR

\end{algorithmic}
\end{algorithm}

\begin{algorithm}
\caption{Resultado de \textit{loop interchanging} no algoritmo~\ref{nf_1}}
\label{nf_2}
\begin{algorithmic}[1]

\FOR {J = 1 to N}
\FOR {I = 1 to N}
\IF {I == J}
\STATE corpo do loop externo
\ELSE
\STATE corpo do loop interno
\ENDIF
\ENDFOR
\ENDFOR

\end{algorithmic}
\end{algorithm}


