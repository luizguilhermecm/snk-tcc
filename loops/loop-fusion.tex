% vim: set ai tw=80 fileencoding=utf8: 
%-------------------------------------------------------------------------------

\subsection{Loop Fusion}

Se dois \textit{loops} são adjacentes e possuem o mesmo limite, algumas vezes
podem ser agrupados em um \textit{loop} simples.
Originalmente, esta técnica fora desenvolvida como uma forma de reduzir os
custos de testes e ramificações no código \cite{Wolf:1996}.

Sem os conceitos de dependências de dados, as descrições anteriores de
\textit{loop fusion} eram restritas a \textit{loops} livres de dependências de
dados. 

O desenvolvimento de \textit{deep memory hierarchies} fez com que esta técnica
fosse importante para aproveitar o \textit{memory locality}.
Utilizando \textit{loop fusion} em \textit{loops} que referenciam os mesmos
dados melhoram o \textit{temporal locality}, podendo assim impactar 
significativamente o desempenho da memória \textit{cache} e da memória virtual.
Outra vantagem do uso de \textit{loop fusion} é tirar vantagem de otimizações 
escalares mais eficiêntes no corpo do \textit{loop}, uma vez que o mesmo ficou
maior após o seu uso \cite{McKinley:1996}.

No algoritmo~\ref{fusion_ex} tem-se dois \textit{loops} adjacentes com limites
compatíveis. 
O algoritmo~\ref{fusion_ex2} mostra o resultado da aplicação do 
\textit{loop fusion}.

\begin{algorithm}
\caption{Algoritmo com dois \textit{loops} adjacentes}
\label{fusion_ex}
\begin{algorithmic}[1]

\FOR {I = 0 to N}
\STATE corpo do loop 1
\ENDFOR
\FOR {I = 0 to N}
\STATE corpo do loop 2
\ENDFOR

\end{algorithmic}
\end{algorithm}

\begin{algorithm}
\caption{Algoritmo~\ref{fusion_ex} após \textit{loop fusion}}
\label{fusion_ex2}
\begin{algorithmic}[1]

\FOR {I = 0 to N}
\STATE corpo do loop 1
\STATE corpo do loop 2
\ENDFOR

\end{algorithmic}
\end{algorithm}

O seu uso é legal se toda dependência de dados for preservada. 
Após a aplicação do \textit{loop fusion} todas as relações de dependências devem 
seguir o fluxo original do "corpo do \textit{loop} 1" para o "corpo do loop
2".

Em alguns casos unir \textit{loops} adjacentes podem ocasionar em violação na
dependência de dados, se um \textit{loop} tem dependência sobre os dados do
outro. Considere os \textit{loops} do algoritmo~\ref{fusion_dep}.

\begin{algorithm}
\caption{Algoritmo com três \textit{loops} adjacentes}
\label{fusion_dep}
\begin{algorithmic}[1]

\FOR {I = 0 to N}
\STATE A(I) = B(I) + 1
\ENDFOR
\FOR {I = 0 to N}
\STATE C(I) = A(I) / 2
\ENDFOR
\FOR {I = 0 to N}
\STATE D(I) = 1 / C(I + 1)
\ENDFOR

\end{algorithmic}
\end{algorithm}

Uma aplicação ingênua do \textit{loop fusion} no algoritmo~\ref{fusion_dep} 
resultaria no algoritmo~\ref{fusion_dep2} que viola as regras básicas de 
dependência.
Considerando o algoritmo~\ref{fusion_dep} existem dois tipos dependência 
de dados, sendo uma depêndencia do tipo \textit{true dependence} ou 
\textit{flow dependence} na relação $2 \; \delta^f \; 5$ e outra entre 
$5 \; \delta^f \; 8$, o segundo tipo de dependência é do tipo
\textit{anti-dependence} na relação $8 \; \delta^a \; 5$, essa dependência
fica mais evidente quando observado o algoritmo~\ref{fusion_dep2}, 
uma vez que a união dos \textit{loops} causa a violação desta dependência. 

Pode-se então observar que o uso não adequado do \textit{loop fusion} violaria
as regras de dependências e assim alterando o sentido do programa. 
Uma aplicação correta da técnica no algoritmo~\ref{fusion_dep} e seu 
resultado pode ser observado no algoritmo~\ref{fusion_dep3}.

\begin{algorithm}
\caption{Violação da dependência de dados do algoritmo~\ref{fusion_dep} após
        \textit{loop fusion}}
\label{fusion_dep2}
\begin{algorithmic}[1]

\FOR {I = 0 to N}
\STATE A(I) = B(I) + 1
\STATE C(I) = A(I) / 2
\STATE D(I) = 1 / C(I + 1)
\ENDFOR

\end{algorithmic}
\end{algorithm}

\begin{algorithm}
\caption{Resultado de uma correta aplicação de \textit{loop fusion} no 
        algoritmo~\ref{fusion_dep}} 
\label{fusion_dep3}
\begin{algorithmic}[1]

\FOR {I = 0 to N}
\STATE A(I) = B(I) + 1
\STATE C(I) = A(I) / 2
\ENDFOR
\FOR {I = 0 to N}
\STATE D(I) = 1 / C(I + 1)
\ENDFOR

\end{algorithmic}
\end{algorithm}

%-------------------------------------------------------------------------------

\subsubsection{Complicações}

Uma vez que \textit{loop fusion} requer que dois \textit{loops} sejam
compatíveis um com o outro para, ou seja, tenham o mesmo número de iterações,
sejam adjacentes e tenham a mesma varíavel de iteração.
Todas estas condições devem ser satisfeitas para que possa ser aplicado o
\textit{loop fusion}.

No caso de algumas das condições ou todas não estejam satisfeitas mas ainda
assim deseja-se aplicar o \textit{loop fusion}, então será necessária a
utilização de transformações nos \textit{loops} para que fiquem compatíveis.

Para \textit{loops} com diferentes limites de percurso, deve-se considerar a
aplicação de \textit{loop peeling} ou \textit{index set splitting} para arrajar
os limites entre eles ou mesmo incluir declarações condicionais para protejer o
código em algumas iterações.
Ambas as alternativas possuem desvantagens, se utilizado \textit{loop peeling}
ou \textit{index set splitting} irá adicionar mais código com um \textit{loop}
extra, mesmo que pequeno. Se adicionar uma declaração condicional dentro do
\textit{loop} deve-se considerar o \textit{overhead} sobre o teste condicional
\cite{Rivera:1999}.

Se os \textit{loops} não forem adjacentes, será então necessário reordenar as
delcarações para que se tornem adjacentes.

Uma complicação extra pode ocorrer caso existam escalares atribuídos em algum
dos \textit{loops}, assim será necessário garantir que após o \textit{loop}
possuam o valor correto.

Considerando o algoritmo~\ref{compl_ex1}, nota-se que os \textit{loops} já são
adjacentes e também são contáveis, mas o primeiro \textit{loop} possui uma
iteração a mais que o segundo. Para corrigir essa iteração a mais pode ser
retirado uma das iterações do primeiro \textit{loop} utilizando 
\textit{loop peeling}, assim teríamos como resultado o algoritmo~\ref{compl_ex2}. 
Para aplicar \textit{loop fusion} no algoritmo~\ref{compl_ex2} deverá ser
criada uma variável $X$ que controle devidamente o valor de $I$, que será utilizada
como variável de controle do \textit{loop}. Como resultado tem-se o
algoritmo~\ref{compl_ex3}.

\begin{algorithm}
\caption{Algoritmo com dois loops adjacentes e contáveis}
\label{compl_ex1}
\begin{algorithmic}[1]

\FOR {I = 1 to 99} 
\STATE A(I) = B(I) + 1
\ENDFOR
\FOR {I = 1 to 98}
\STATE C(I) = A[I + 1] * 2
\ENDFOR

\end{algorithmic}
\end{algorithm}

\begin{algorithm}
\caption{Resultado de \textit{loop peeling} no algoritmo~\ref{compl_ex1}}
\label{compl_ex2}
\begin{algorithmic}[1]

\STATE I = 1
\STATE A(I) = B(I) + 1
\FOR {I = 2 to 98} 
\STATE A(I) = B(I) + 1
\ENDFOR
\FOR {I = 1 to 98}
\STATE C(I) = A[I + 1] * 2
\ENDFOR

\end{algorithmic}
\end{algorithm}

\begin{algorithm}
\caption{Resultado de \textit{loop fusion} no algoritmo~\ref{compl_ex1}}
\label{compl_ex3}
\begin{algorithmic}[1]

\STATE I = 1
\STATE A(I) = B(I) + 1
\FOR {X = 0 to 97} 
\STATE I = X + 2
\STATE A(I) = B(I) + 1
\STATE I = X + 1
\STATE C(I) = A[I + 1] * 2
\ENDFOR

\end{algorithmic}
\end{algorithm}

Um exemplo de \textit{loops} que não se pode aplicar \textit{loop fusion} pode
ser visto no algoritmo~\ref{compl__}, uma vez que o escalar $D$ é atribuído no
primeiro \textit{loop} e utilizado no segundo.

\begin{algorithm}
\caption{Exemplo de \textit{loops} que não podem ser unidos}
\label{compl__}
\begin{algorithmic}[1]

\FOR {I = 0 to 100} 
\STATE D = A(I)
\STATE B(I) = D
\ENDFOR
\FOR {J = 0 to 100} 
\STATE C(J) = C(J) + D
\ENDFOR

\end{algorithmic}
\end{algorithm}
