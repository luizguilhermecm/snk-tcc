% vim: set ai tw=80 fileencoding=utf8: 
%-------------------------------------------------------------------------------

\section{Loop Fusion}

Se dois \textit{loops} são adjacentes e possuem o mesmo limite, algumas vezes
podem ser agrupados em um \textit{loop} simples.
Originalmente, esta técnica fora desenvolvida como uma forma de reduzir os
custos de testes e ramificações no código.

Sem os conceitos de dependências de dados, as descrições anteriores de
\textit{loop fusion} eram restritas a \textit{loops} livres de dependências de
dados. 

O desenvolvimento de \textit{deep memory hierarchies} fez com que esta técnica
fosse importante para aproveitar o \textit{memory locality}.
Utilizando \textit{loop fusion} em \textit{loops} que referenciam os mesmos
dados melhoram o \textit{temporal locality}, podendo assim impactar 
significativamente o desempenho da memória \textit{cache} e da memória virtual.
Outra vantagem do uso de \textit{loop fusion} é tirar vantagem de otimizações 
escalares mais eficiêntes no corpo do \textit{loop}, uma vez que o mesmo ficou
maior após o seu uso.

No algoritmo ~\ref{fusion_ex} tem-se dois \textit{loops} adjacentes com limites
compatíveis. 
O algoritmo ~\ref{fusion_ex2} mostra o resultado da aplicação do 
\textit{loop fusion}.

\begin{algorithm}
\caption{Algoritmo com dois \textit{loops} adjacentes}
\label{fusion_ex}
\begin{algorithmic}[1]

\FOR {I = 0 to N}
\STATE corpo do loop 1
\ENDFOR
\FOR {I = 0 to N}
\STATE corpo do loop 2
\ENDFOR

\end{algorithmic}
\end{algorithm}

\begin{algorithm}
\caption{Algoritmo ~\ref{fusion_ex} após \textit{loop fusion}}
\label{fusion_ex2}
\begin{algorithmic}[1]

\FOR {I = 0 to N}
\STATE corpo do loop 1
\STATE corpo do loop 2
\ENDFOR

\end{algorithmic}
\end{algorithm}

O seu uso é legal se toda dependência de dados for preservada. 
Após a aplicação do \textit{loop fusion} todas as relações de dependências devem 
seguir o fluxo original do "corpo do \textit{loop} 1" para o "corpo do loop
2".

Em alguns casos unir \textit{loops} adjacentes podem ocasionar em violação na
dependência de dados, se um \textit{loop} tem dependência sobre os dados do
outro. Considere os \textit{loops} do algoritmo ~\ref{fusion_dep}.

\begin{algorithm}
\caption{Algoritmo com três \textit{loops} adjacentes}
\label{fusion_dep}
\begin{algorithmic}[1]

\FOR {I = 0 to N}
\STATE A(I) = B(I) + 1
\ENDFOR
\FOR {I = 0 to N}
\STATE C(I) = A(I) / 2
\ENDFOR
\FOR {I = 0 to N}
\STATE D(I) = 1 / C(I + 1)
\ENDFOR

\end{algorithmic}
\end{algorithm}

Uma aplicação ingênua do \textit{loop fusion} no algoritmo ~\ref{fusion_dep} 
resultaria no algoritmo ~\ref{fusion_dep2} que viola as regras básicas de 
dependência.
Considerando o algoritmo ~\ref{fusion_dep} existem dois tipos dependência 
de dados, sendo uma depêndencia do tipo \textit{true dependence} ou 
\textit{flow dependence} na relação $2 \; \delta^f \; 5$ e outra entre 
$5 \; \delta^f \; 8$, o segundo tipo de dependência é do tipo
\textit{anti-dependence} na relação $8 \; \delta^a \; 5$, essa dependência
fica mais evidente quando observado o algoritmo ~\ref{fusion_dep2}, 
uma vez que a união dos \textit{loops} causa a violação desta dependência. 

Pode-se então observar que o uso não adequado do \textit{loop fusion} violaria
as regras de dependências e assim alterando o sentido do programa. 
Uma aplicação correta da técnica no algoritmo ~\ref{fusion_dep} e seu 
resultado pode ser observado no algoritmo ~\ref{fusion_dep3}.

\begin{algorithm}
\caption{Violação da dependência de dados do algoritmo ~\ref{fusion_dep} após
        \textit{loop fusion}}
\label{fusion_dep2}
\begin{algorithmic}[1]

\FOR {I = 0 to N}
\STATE A(I) = B(I) + 1
\STATE C(I) = A(I) / 2
\STATE D(I) = 1 / C(I + 1)
\ENDFOR

\end{algorithmic}
\end{algorithm}

\begin{algorithm}
\caption{Resultado de uma correta aplicação de \textit{loop fusion} no 
        algoritmo ~\ref{fusion_dep}} 
\label{fusion_dep3}
\begin{algorithmic}[1]

\FOR {I = 0 to N}
\STATE A(I) = B(I) + 1
\STATE C(I) = A(I) / 2
\ENDFOR
\FOR {I = 0 to N}
\STATE D(I) = 1 / C(I + 1)
\ENDFOR

\end{algorithmic}
\end{algorithm}

%-------------------------------------------------------------------------------

\subsection{Array Assignments}


