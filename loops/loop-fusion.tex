% vim: set ai tw=80 fileencoding=utf8: 
%-------------------------------------------------------------------------------

\section{Loop Fusion}

Se dois \textit{loops} são adjacentes e possuem o mesmo limite, estes podem
ser agrupados em um \textit{loop} simples.
Originalmente, esta técnica fora desenvolvida como uma forma de reduzir os
custos de testes e ramificações no código.

Sem os conceitos de dependências de dados, as descrições anteriores de
\textit{loop fusion} eram restritas a \textit{loops} livres de dependências de
dados. 

O desenvolvimento de \textit{deep memory hierarchies} fez com que esta técnica
fosse importante para aproveitar o \textit{memory locality}.
Utilizando \textit{loop fusion} em \textit{loops} que referenciam os mesmos
dados melhoram o \textit{temporal locality}, podendo impactar significativamente  
o desempenho da memória \textit{cache} e da memória virtual.
Outra vantagem do uso de \textit{loop fusion} seria tirar vantagem de otimizações 
escalares mais eficiêntes no corpo do \textit{loop}, uma vez que o mesmo ficou
maior após o \textit{loop fusion}.

O uso desta técnica é legal se toda dependência de dados é preservada. Após a
aplicação do \textit{loop fusion} todas as relações de dependências devem seguir
o fluxo original.



