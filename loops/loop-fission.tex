% vim: set ai tw=80 fileencoding=utf8: 
%-------------------------------------------------------------------------------

\subsection{Loop Fission}

Um \textit{loop} simples pode ser quebrado em dois ou mais \textit{loops}
menores, isso é o inverso de \textit{loop fusion}, essa técnica é chamada de
\textit{loop fission} ou \textit{loop distribution}.
\textit{Loop fission} é muito utilizada em máquinas com 
\textit{instruction cache} menores, esta técnica pode ser utilizada para quebrar
grandes \textit{loops} que não seriam suportados pela \textit{cache} em
\textit{loops} menores que caibam na \textit{cache}.
Podendo também melhorar a \textit{memory locality} em casos onde um único
\textit{loop} utiliza vários vetores, assim dividindo estes vetores entre mains 
de um \textit{loop}.

Quando um \textit{loop} é dividido, o \textit{index} ou \textit{induction
variable} será utilizado em cada parte do \textit{loop}. 
Para evitar problemas em saber quando a \textit{induction variable} será
atualizada, pode ser criada uma nova variável que a substitua, assim trocando
todas as referências por esta nova variável.

Antes de aplicar esta técnica, primeiro será construído um grafo de dependência em
nível das declarações do \textit{loop}. 
Caso não haja ciclos no grafo de dependência, então o \textit{loop} poderá ser
divido em cada nó do grafo. 
Em caso de ciclos no grafo, então a abordagem não poderá ser a mesma, pois
violaria a dependência de dados, logo, se dois nós formam um ciclo no grafo,
eles podem ser unidos em um único nó, pois não será possível separa-los.
Se o grafo de dependência for fortemente conectado, então a não será possível a
aplicação de \textit{loop fission}.

A ordem correta dos \textit{loops} após o \textit{loop fission} pode ser
encontrada através de um grafo acíclico condensado do grafo de dependência.
Sendo cada nó no grafo condensado um \textit{loop}.

Considerando o algoritmo~\ref{fission_ex1} com quatro declarações em seu corpo e
quatro vetores. 
O grafo de dependência pode ser visto na figura~\ref{graph_fission_ex1}. 
Como pode-se observar no grafo de dependência, existe um ciclo entre as linhas
$3$ e $4$, sendo assim, não será possível separa-las em diferentes
\textit{loops}, também existe uma dependência entre a linha $2$ com a linha $3$
e da linha $5$ com a $4$.
O grafo condensado na figura~\ref{condensed_graph_fission_ex1} mostra a
ordem em que os \textit{loops} devem estar.
O algoritmo~\ref{fission_ex2} mostra o resultado de \textit{loop fission} sobre
o algoritmo~\ref{fission_ex1}.

\begin{algorithm}
\caption{Algoritmo em que ser quer aplicar \textit{loop fission}}
\label{fission_ex1}
\begin{algorithmic}[1]

\FOR {I = 1 to N} 
\STATE A(I) = A(I) + B(I - 1)
\STATE B(I) = C(I - 1) * X + C
\STATE C(I) = 1 / B(I)
\STATE D(I) = sqrt(C(I))
\ENDFOR

\end{algorithmic}
\end{algorithm}

\begin{figure}
\centering
\label{graph_fission_ex1}
\begin{tikzpicture}[->,>=stealth',shorten >=1pt,auto,node distance=1.5cm,
  thick,main node/.style={circle,draw}]

  \node[main node] (2) {2};
  \node[main node] (3) [below of=2] {3};
  \node[main node] (4) [below of=3] {4};
  \node[main node] (5) [below of=4] {5};

   \path[every node/.style={font=\sffamily\small}]
    (3) edge [bend left] node [below] {} (2)
        edge node [above] {} (4)
    (4) edge [bend left] node [below] {} (3)
        edge node [above] {} (5);

\end{tikzpicture}
\caption{Grafo de dependência do algoritmo~\ref{fission_ex1}}
\end{figure}

\begin{figure}
\centering
\label{condensed_graph_fission_ex1}
\begin{tikzpicture}[->,>=stealth',shorten >=1pt,auto,node distance=1.5cm,
  thick,main node/.style={circle,draw}]

  \node[main node] (3) {3-4};
  \node[main node] (2) [below left of=3] {2};
  \node[main node] (5) [below right of=3] {5};

   \path[every node/.style={font=\sffamily\small}]
    (3) edge node [below left] {} (2)
        edge node [below right] {} (5);

\end{tikzpicture}
\caption{Grafo condensado acíclico do grafo de dependência}
\end{figure}

\begin{algorithm}
\caption{Resultado de \textit{loop fission} sobre o algoritmo~\ref{fission_ex1}}
\label{fission_ex2}
\begin{algorithmic}[1]

\FOR {I = 0 to N - 1} 
\STATE B(I + 1) = C(I) * X + C
\STATE C(I + 1) = 1 / B(I + 1)
\ENDFOR
\FOR {I = 0 to N - 1} 
\STATE A(I + 1) = A(I + 1) + B(I)
\ENDFOR
\FOR {I = 0 to N - 1} 
\STATE D(I + 1) = sqrt(C(I + 1))
\ENDFOR
\STATE I = I + 1
\end{algorithmic}
\end{algorithm}


