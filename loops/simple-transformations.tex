% vim: set ai tw=80 fileencoding=utf8: 
%-------------------------------------------------------------------------------

\section{Transformações Simples}

Serão apresentadas algumas transformações simples em \textit{loops} que também
serão utilizadas em outras técnicas apresentadas neste capítulo.

%-------------------------------------------------------------------------------
\subsection{Reordenação das Declarações}

A reordenação pode ser realizada sobre qualquer granularidade, operação,
declaração ou sequência de declarações.

Está técnica pode ser utilizada para vários propósitos. Sendo utilizanda para 
amortizar a latência da memória, melhorar o \textit{data locality} movendo 
declarações ou \textit{loops} que utilizam as mesmas variáveis próximas uma das 
outras e também pode ser utilizada para mover \textit{loops} que estão separados 
para perto um do outro, talvéz permitindo o uso da técnica \textit{loop fusion} 
ou outra técnica.




%-------------------------------------------------------------------------------
\subsection{Unswitching}

\textit{Unswitching} é uma transformação simples que retira do \textit{loop} 
a condição independe do \textit{loop}. 
\textit{Unswitching} trata um \textit{loop} que contenha uma condição e faz com 
que a condição envolva o \textit{loop}. 
A condição deve obrigatoriamente ser independente do \textit{loop}.
A vantagem do uso do textit{unswitching} é reduzir a frequência de execução da 
condição, uma vez que fora retirada do \textit{loop}. 
A desvantagem desta tranformação é o aumento da complexidade da estrutura do 
\textit{loop}, um \textit{loop} contendo apenas um \textit{loop} mais interno,
agora terá dois ou mais \textit{loops} internos. 
Essa desvantagem pode afetar a aplicabilidade de outras técnicas ou
transformações.




%-------------------------------------------------------------------------------
\subsection{Loop Peeling}

\textit{Loop peeling} remove a primeira ou a ultima iteração do \textit{loop} 
em código separado, podendo ser generalizado e realizado mais de uma vez. 
Essa técnica depente de saber se a variável de controle da iteração é estritamente
positiva, caso não o seja deve-se realizar os devidos cuidados. 
\textit{Loop peeling} também pode ser utilizado para retirar partes de código 
que não dependem do \textit{index} do \textit{loop}, assim, executando-os apenas
uma vez.


%-------------------------------------------------------------------------------
\subsection{Index Set Splitting}

\textit{Index set splitting} ou \textit{loop splitting} é uma generalização de 
\textit{loop peeling}. Essa técnica divide o \textit{index set} de um
\textit{loop} em duas partes, replicando o corpo do \textit{loop}
apropriadamente. 
Assim como em \textit{loop peeling} essa técnica é útil para ajudar o percurso
do \textit{loop}, ou para remover condições que testam a variável que controla
as iterações do \textit{loop}.



%-------------------------------------------------------------------------------
\subsection{Scalar Expansion} 

\ldots
