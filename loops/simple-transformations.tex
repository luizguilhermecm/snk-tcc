% vim: set ai tw=80 fileencoding=utf8: 
%-------------------------------------------------------------------------------

\subsection{Transformações Simples}

Serão apresentadas algumas transformações simples em \textit{loops} que também
serão utilizadas em outras técnicas apresentadas nesta seção.

%-------------------------------------------------------------------------------
\subsubsection{Reordenação de Declarações}

A reordenação pode ser realizada sobre qualquer granularidade, operação,
declaração ou sequência de declarações. 
Será utilizado um grafo de controle aciclico (CFG) ou grafo de dependência.
Um nó do grafo pode representar uma declaração ou um \textit{loop} inteiro.
A reordenação é considerada válida desde que respeite a dependência de dados, 
ou seja, não altere o significado do programa \cite[9.1]{hpcfpc}.

Está técnica pode ser utilizada para vários propósitos, para 
amortizar a latência da memória, melhorar o \textit{data locality} movendo 
declarações ou \textit{loops} que utilizam as mesmas variáveis próximas uma das 
outras. 
A reordenação de declarações é útil pois através dela é possível 
a aplicação de outras técnicas. 

Utilizando o algoritmo~\ref{reordenacao} para exemplificar a reordenação. 

\begin{algorithm}
\caption{Exemplo de algoritmo a ser reordenado}
\label{reordenacao}
\begin{algorithmic}[1]

\STATE A(1) = 0
\STATE B(1) = 0
\IF {C > 0}
\STATE A(2) = 1
\STATE B(2) = 9
\ENDIF
\FOR {I = 3 to 9}
\STATE A(I) = A(I - 2) + A(I - 1) * 2
\STATE B(I) = B(I - 2) * 2 + B(I - 1) 
\ENDFOR

\end{algorithmic}
\end{algorithm}

O grafo de dependência do algoritmo~\ref{reordenacao} pode ser visto na 
figura~\ref{graph_statement_reordering}. 
Neste exemplo deseja-se que as declarações $1$ e $4$ fiquem o mais próximo 
possível do \textit{loop}. 
Uma forma de faze-lo é reornando as declarações para a seguinte ordem $2,\; 3,\; 5,
\;1,\; 4\; e\; 7$, como resultado tem-se o algoritmo~\ref{reordenado}.



\begin{figure}
\centering
\label{graph_statement_reordering}
\begin{tikzpicture}[->,>=stealth',shorten >=1pt,auto,node distance=1.5cm,
  thick,main node/.style={circle,draw}]

  \node[main node] (1) {1};
  \node[main node] (2) [below of=1] {2};
  \node[main node] (3) [below of=2] {3};
  \node[main node] (4) [below of=3] {4};
  \node[main node] (5) [below of=4] {5};
  \node[main node] (7) [below of=5] {7};

  \path[every node/.style={font=\sffamily\small}]
    (1) edge [bend right] node [above] {} (4)
        edge [bend right] node [above] {} (7)
    (2) edge [bend left] node [above] {} (5)
        edge [bend left] node [above] {} (7)
    (3) edge node [above] {} (4)
        edge [bend right] node [above] {} (5)
    (4) edge [bend left] node [above] {} (7)
    (5) edge node [above] {} (7);

\end{tikzpicture}
\caption{Grafo de dependência do algoritmo~\ref{reordenacao}}
\end{figure}

\begin{algorithm}
\caption{Algoritmo~\ref{reordenacao} após a reordenação}
\label{reordenado}
\begin{algorithmic}[1]

\STATE B(1) = 0
\STATE Test = C > 0
\IF {Test} \STATE B(2) = 9 \ENDIF
\STATE A(1) = 0
\IF {Test} \STATE A(2) = 1 \ENDIF
\FOR {I = 3 to 9}
\STATE A(I) = A(I - 2) + A(I - 1) * 2
\STATE B(I) = B(I - 2) * 2 + B(I - 1) 
\ENDFOR

\end{algorithmic}
\end{algorithm}


%-------------------------------------------------------------------------------
\subsubsection{Unswitching}

\textit{Unswitching} é uma transformação simples que retira do \textit{loop} 
uma condição que independe do \textit{loop}. 
\textit{Unswitching} trata um \textit{loop} que tenha uma condição e faz com 
que a condição envolva um ou dois \textit{loops}. 
A condição deve obrigatoriamente ser independente do \textit{loop}.
A vantagem do uso do \textit{unswitching} é a redução da frequência de execução da 
condição, uma vez que fora retirada do \textit{loop}. 
A desvantagem é o aumento da complexidade da estrutura do \textit{loop}, 
um \textit{loop} contendo apenas um \textit{loop} mais interno,
poderá ter dois ou mais \textit{loops} internos. 
Essa desvantagem pode afetar a aplicabilidade de outras técnicas ou
transformações \cite{hpcfpc}.

\textit{Unswitching} pode ser aplicado em um \textit{loop} semelhante ao 
algoritmo~\ref{unswitching_ex}, onde a condição é independente do \textit{loop}.


\begin{algorithm}
\caption{Algoritmo onde a condição independe do \textit{loop}}
\label{unswitching_ex}
\begin{algorithmic}[1]

\LOOP
\STATE declarações
\IF {teste} 
\STATE declarações de then
\ELSE 
\STATE declarações de else
\ENDIF
\STATE mais declarações
\ENDLOOP

\end{algorithmic}
\end{algorithm}

O algoritmo~\ref{unswitching_ex} pode ser transformado no 
algoritmo~\ref{unswitching_ex2}.

\begin{algorithm}
\caption{Algoritmo~\ref{unswitching_ex} depois de aplicar \textit{unswitching}}
\label{unswitching_ex2}
\begin{algorithmic}[1]

\IF {teste} 
\LOOP
\STATE declarações
\STATE declarações de then
\STATE mais declarações
\ENDLOOP
\ELSE 
\LOOP
\STATE declarações
\STATE declarações de else
\STATE mais declarações
\ENDLOOP
\ENDIF

\end{algorithmic}
\end{algorithm}

Considerando o algoritmo~\ref{unswitching_inner} por apresentar um 
\textit{loop} mais interno. 

\begin{algorithm}
\caption{Algoritmo com \textit{loops} aninhados}
\label{unswitching_inner}
\begin{algorithmic}[1]

\FOR {I = 1 to N}
\FOR {J = 2 to N}
\IF {T(I) > 0} 
\STATE A(I,J) = A(I, J - 1) * T(I) + B(J) 
\ELSE
\STATE A(I,J) = 0 0 
\ENDIF
\ENDFOR
\ENDFOR

\end{algorithmic}
\end{algorithm}

Aplicando-se \textit{unswitching} no algoritmo~\ref{unswitching_inner} será 
removida a condição do \textit{loop} mais interno, ver 
algoritmo~\ref{unswitching_inner2}.

\begin{algorithm}
\caption{Algoritmo~\ref{unswitching_inner} depois de \textit{unswitching}}
\label{unswitching_inner2}
\begin{algorithmic}[1]

\FOR {I = 1 to N}
\IF {T(I) > 0} 
\FOR {J = 2 to N}
\STATE A(I,J) = A(I, J - 1) * T(I) + B(J) 
\ENDFOR
\ELSE
\FOR {J = 2 to N}
\STATE A(I,J) = 0 0 
\ENDFOR
\ENDIF
\ENDFOR

\end{algorithmic}
\end{algorithm}

%-------------------------------------------------------------------------------
\subsubsection{Loop Peeling}

\textit{Loop peeling} remove a primeira ou a ultima iteração do \textit{loop} 
colocando-a em código separado, podendo ser generalizado e realizado mais 
de uma vez se necessário. 
Essa técnica depente de saber se a variável de controle da iteração
(\textit{index}) é estritamente positiva, caso não o seja, deve-se proteger 
o código com uma condição maior que zero.
\textit{Loop peeling} também pode ser utilizado para retirar partes de código 
que não dependem do \textit{index} do \textit{loop}, assim, executando-os apenas
uma vez \cite{hpcfpc}.

\textit{Loop peeling} também pode ser utilizado para ajustar a variação do 
\textit{index} do \textit{loop}, assim tornando possível a aplicação 
do \textit{loop fusion}.

Considerando o algoritmo~\ref{peeling_ex} um \textit{loop} contável e 
com \textit{index} tendo como máximo $MAX$.

\begin{algorithm}
\caption{Algoritmo com \textit{loop} contável}
\label{peeling_ex}
\begin{algorithmic}[1]

\STATE calcula MAX
\FOR {I = 0 to MAX - 1}
\STATE corpo do loop
\ENDFOR

\end{algorithmic}
\end{algorithm}

Aplicando \textit{loop peeling} no algoritmo~\ref{peeling_ex} um resultado
possível seria o algoritmo~\ref{peeling_ex2}.

\begin{algorithm}
\caption{Algoritmo~\ref{peeling_ex} depois de \textit{loop peeling}}
\label{peeling_ex2}
\begin{algorithmic}[1]

\STATE calcula MAX
\IF {MAX > 0}
\STATE I = 0
\STATE corpo do loop
\FOR {I = 1 to MAX - 1}
\STATE corpo do loop
\ENDFOR
\ENDIF

\end{algorithmic}
\end{algorithm}


%-------------------------------------------------------------------------------
\subsubsection{Index Set Splitting}

\textit{Index set splitting} ou \textit{loop splitting} é uma generalização de 
\textit{loop peeling}. Essa técnica divide o \textit{index set} de um
\textit{loop} em duas partes, replicando o corpo do \textit{loop}
apropriadamente. 
Assim como em \textit{loop peeling} essa técnica é útil para ajustar o percurso
do \textit{loop}, ou para remover condições que testam a variável que controla
as iterações do \textit{loop} \cite{hpcfpc}.

Considerando o algoritmo~\ref{iss_ex} com um \textit{loop} com percurso indo
até $MAX$.

\begin{algorithm}
\caption{Algoritmo com um \textit{loop} de percurso até $MAX$}
\label{iss_ex}
\begin{algorithmic}[1]

\STATE calcula MAX
\FOR {I = 0 to MAX - 1}
\STATE corpo do loop
\ENDFOR

\end{algorithmic}
\end{algorithm}

Aplicando \textit{index set splitting} em~\ref{iss_ex} na iteração $W$
resultaria no algoritmo~\ref{iss_ex2}.

\begin{algorithm}
\caption{Algoritmo~\ref{iss_ex} depois de \textit{index set splitting}}
\label{iss_ex2}
\begin{algorithmic}[1]

\STATE calcula MAX
\FOR {I = 0 to W - 1}
\STATE corpo do loop
\ENDFOR
\FOR {I = S to MAX - 1}
\STATE corpo do loop
\ENDFOR

\end{algorithmic}
\end{algorithm}


%-------------------------------------------------------------------------------
\subsubsection{Scalar Expansion} 

Uma vez que escalares são atribuídos e depois utilizados no \textit{loop}, o 
grafo de dependência irá incluir relações de dependência de fluxo da atribuição 
para cada uso, o \textit{loop} terá relações de \textit{anti-dependence}.
Esta dependência pode causar problemas ao tentar aplicar outras transformações.
As relações de dependências podem ser quebradas através de \textit{expanding} ou 
\textit{promoting}, colocando o escalar em um vetor.
Considerando apenas \textit{loops} contáveis e com o escalar não tendo uso
anterior ao loop. Neste caso alocando-se um vetor com o tamanho do percurso do
\textit{loop} e trocando a referência de cada iteração para sua respectiva
posição no vetor irá satisfazer as relações de dependências. Com cada iteração
do \textit{loop} utilizando uma posição do vetor a \textit{anti-dependence} e 
\textit{output dependence} são eliminadas \cite{hpcfpc}.

O uso de \textit{scalar expansion} em \textit{loops} aninhados é possível,
embora o tamanho do percurso do \textit{loop} possa ser proibitivo.

Após o final do \textit{loop} deve ser atribuído ao escalar o seu valor correto,
ou se constatado que o escalar não mais será utilizado após o \textit{loop},
então basta desconsidera-lo.
Caso o valor do escalar seja determinado condicionalmente dentro do
\textit{loop} invibializaria o uso de \textit{scalar expansion} exceto se o
mesmo não será mais utilizado após o \textit{loop}.

Tendo o algoritmo~\ref{se_ex} com escalar $E$.


\begin{algorithm}
\caption{Algoritmo contendo um escalar}
\label{se_ex}
\begin{algorithmic}[1]

\FOR {I = 0 to N}
\STATE E = A(I) + B(I)
\STATE C(I) = E + 1/E
\ENDFOR

\end{algorithmic}
\end{algorithm}

Aplicando \textit{scalar expansion} no algoritmo~\ref{se_ex} e protegendo-o com
uma condição maior que zero, irá eliminar todas as relações de dependências, o
resultado é o algoritmo~\ref{se_ex2}

\begin{algorithm}
\caption{Algoritmo~\ref{se_ex} após o \textit{scalar expansion}}
\label{se_ex2}
\begin{algorithmic}[1]

\IF {N >= 0}
\STATE alloc Ex(1:N)
\FOR {I = 0 to N}
\STATE Ex(I) = A(I) + B(I)
\STATE C(I) = Ex(I) + 1/Ex(I)
\ENDFOR
\STATE E = Ex(N)
\ENDIF

\end{algorithmic}
\end{algorithm}




%-------------------------------------------------------------------------------
\subsubsection{Loop Unrolling} 

A técnica \textit{loop unrolling} baseia-se em duplicar o corpo do \textit{loop} 
$n$ vezes, fazendo com que o número de iterações seja menor, no caso $n$ vezes
menor.
Com isso, aproveitando possíveis melhorias que possam existir durante as
iterações, entre elas a diminuição do número de vezes em que será necessário
realizar a comparação da condição de saída do \textit{loop}.
Com a duplicação do corpo do \textit{loop} e com isso o incremento a cada
iteração, se o valor máximo de iteração não for par deverá então ser adicionando
um tratamento para evitar que a ultima iteração seja realizada
\cite{Dragomir:2009}.

\textit{Loop unrolling} irá aumentar o número de linhas do código com isso 
dificultando a leitura do mesmo.
Considerando o algoritmo~\ref{unroll_ex} e aplicando \textit{loop unrolling}
apenas duplicando uma vez, 
tem-se como resultado o algoritmo~\ref{unroll_ex1}.

\begin{algorithm}
\caption{Loop simples com N par}
\label{unroll_ex}
\begin{algorithmic}[1]

\FOR {I = 0 to N}
\STATE A(I) = B(I)
\ENDFOR

\end{algorithmic}
\end{algorithm}

\begin{algorithm}
        \caption{Algoritmo~\ref{unroll_ex} após \textit{loop unrolling}}
\label{unroll_ex1}
\begin{algorithmic}[1]

\FOR {I = 0 to N by 2}
\STATE A(I) = B(I)
\STATE A(I + 1) = B(I + 1)
\ENDFOR

\end{algorithmic}
\end{algorithm}
